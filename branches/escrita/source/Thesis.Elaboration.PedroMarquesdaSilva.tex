%% Template para dissertao/tese na classe thesisdifctunl  
%%  EM  DESENVOLVIMENTO - Dezembro de 2007
%%
%% Carrega a classe thesisdifctunlpt
%% Opes: 
%% * Idiomas
%%           pt   - portugus 
%%           en   - ingls
%% * Tipo de texto
%%           lei  - licenciatura
%%           premei - preparao de dissertao
%%           mei  - mestrado
%%           prop - proposta de doutoramento
%%           prephd - preparao de tese
%%           phd  - doutoramento
%%   * Tipo de impresso
%%            print
%%    * Pginas
%%           oneside - para impresso em face nica
%%           twoside - para impresso em frente e verso
\documentclass[pt, mei, twoside, print]{thesisdifctunl}

%% Prembulo:
%% coloque aqui o seu prembulo LaTeX, i.e., declarao de pacotes,
%% (re)definies de macros, medidas, etc. 
\newcommand{\descvspace}{\addtolength{\itemsep}{-.5\baselineskip}}
\newcommand{\mc}[3]{\multicolumn{#1}{#2}{#3}}
\newcommand{\sups}[1]{\ensuremath{^{\textrm{#1}}}}
\newcommand{\subs}[1]{\ensuremath{_{\textrm{#1}}}}
\newenvironment{FramedVerb}%
{\scriptsize\VerbatimEnvironment
\begin{Sbox}\begin{minipage}{15cm}\begin{Verbatim}}%
{\end{Verbatim}\end{minipage}\end{Sbox}
\setlength{\fboxsep}{8pt}\fbox{\TheSbox}\normalsize }
%\usepackage{tweaklist}
% \usepackage{algorithm}
\usepackage{algpseudocode}
% \usepackage{algorithm2e}    
\usepackage{listings}
\usepackage{fancybox}
\usepackage{setspace}
\usepackage{multirow}
\usepackage[table]{xcolor}
\usepackage{float}
\usepackage{color}
\usepackage{colortbl}
\usepackage{rotating}
\usepackage{enumitem}
\usepackage{hyperref}
\hypersetup{
    bookmarks=true,         % show bookmarks bar
    unicode=false,          % non-Latin characters in Acrobats bookmarks
    pdftoolbar=true,        % show Acrobat toolbar
    pdfmenubar=true,        % show Acrobat menu
    pdffitwindow=false,     % window fit to page when opened
    pdfstartview={FitH},    % fits the width of the page to the window
    pdftitle={Avalia��o de condi��es de fiabilidade e seguran�a de protocolos de encaminhamento de
dados em redes de sensores sem fios},    % title
    pdfauthor={Pedro Miguel Oliveira Marques da Silva},     % author
    pdfsubject={Masters in Science in Computer Engeneering},   % subject of the document
    pdfcreator={Latex},   % creator of the document
    pdfproducer={Faculdade de Ci�ncias e Tecnologia, UNL}, % producer of the document
    pdfkeywords={Wireless Sensor Networks, Secure Routing Protocols, WSN Simulation, Intrusion
Atack}, % list of keywords
    pdfnewwindow=true,      % links in new window
    colorlinks=true,       % false boxed links true: colored links
    linkcolor=black,          % color of internal links
    citecolor=black,        % color of links to bibliography
    filecolor=black,      % color of file links
    urlcolor=black,           % color of external links
		pagebackref=true
}
\setcounter{tocdepth}{2}

\newenvironment{descriptionNoIndent}
{
  \begin{description}[leftmargin=0mm,topsep=5mm]
}
{
  \end{description}
}
\newenvironment{descriptionNoIndent2}
{
  \begin{description}[leftmargin=0mm,topsep=3mm,noitemsep]
}
{
  \end{description}
}
%\setlength{\parskip}{0mm plus0mm minus0mm}
%% Identificao:
% Universidade
\university{Universidade Nova de Lisboa}
% Endereo (cidade).
\address{Lisboa}
% Faculdade
\faculty{Faculdade de Ci�ncias e Tecnologia}
% Departamento 
\department{Departamento de Inform�tica}
% Curso
\program{Mestrado em Engenharia Inform�tica}
% rea cientfica
\majorfield{Engenharia Inform�tica}
% Ttulo da dissertao/tese
\title{Avalia��o de condi��es de fiabilidade e seguran�a de protocolos de encaminhamento de dados em
redes de sensores sem fios}

% Data
\date{1� Semestre de 2009/10 \\ \smallskip
5 de Fevereiro de 2010}

% Autor
\author{Pedro Miguel Oliveira Marques da Silva (n� 26649)}

% Orientador(a)
% Opo: [f] - para orientador do sexo feminino
\adviser{Prof. Doutor Henrique Jo�o Lopes Domingos}

% Co-orientador(a)
%\coadviser{Prof. Doutor nome do co-orientador}

%% Inicio do documento
\begin{document}
\onehalfspacing
%% Parte pr-textual
\frontmatter

% Folha de rosto. Comente para ocultar
\frontpage

% Pgina de apresentao. Comente para ocultar
\presentationpage

% Agradecimentos. Se preferir, crie um ficheiro  parte e inclua com \include{}
\acknowledgements

% O texto ...
\contracapa
{
\setlength{\parindent}{0in}
N� do aluno: 	26649
\vskip.2\baselineskip
Nome: 	Pedro Miguel Oliveira Marques da Silva	
\vskip.8\baselineskip
T�tulo da disserta��o: 

Avalia��o de condi��es de fiabilidade e seguran�a de protocolos de
encaminhamento de dados em redes
de sensores sem fios
\vskip.8\baselineskip
Palavras-Chave:
\vskip.5\baselineskip
\begin{itemize}
\item Simula��o de redes de sensores
\item Redes de sensores sem fios
\item Protocolos de encaminhamento seguros
\item Escala
\item Fiabilidade
\item Seguran�a
\item Toler�ncia a Intrus�es


\end{itemize}
\vskip1.0\baselineskip
Keywords:
\vskip.5\baselineskip
\begin{itemize}
\item WSN Simulation
\item Wireless Sensor Networks
\item Secure Routing Protocols
\item Scalability
\item Reliability
\item Security
\item Intrusion Tolerance

\end{itemize}




}
% Resumo em Portugus
% Se preferir, crie um ficheiro  parte e inclua com \include{}
\resumo
As redes de sensores s�o uma tecnologia emergente no dom�nio da monitoriza��o
, de forma aut�noma, de ambientes fisicos. S�o formadas por pequenos
dispositivos que se auto-organizam por modo a cobrirem uma �rea geogr�fica. Esta
autonomia e auto-organiza��o apresenta alguns desafios relacionados com os
aspectos de seguran�a, nomeadamente, no que concerne com o encaminhamento de
dados. Assim, o trabalho a realizar pretende contribuir para a cria��o
de um modelo sist�mico para o estudo de protocolos de encaminhamento seguro em
redes de sensores sem fios (RSSF). A defini��o do modelo de advers�rio � o passo
inicial para o enquadramento das tipologias de ataque que se pretende avaliar.
Aliado ao modelo formal de Dolev-Yao, orientado para os ataques ao meio de
comunica��o, o estudo de novos modelos de advers�rio, relacionados com a
intrus�o ou captura de n�s, quer sejam bizantinos ou probabilisticos, �
pertinente e apresentado dentro do �mbito deste trabalho.

Com vista a tornar as RSSF resistentes a algumas tipologias de ataques
preconizadas no modelo de advers�rio, t�m vindo a ser desenvolvidos diversos
algoritmos de encaminhamento seguro. Pretende-se estudar alguns destes
algoritmos, representantes do estado da arte neste dom�nio, estabelecendo uma
matriz de medidas de resist�ncia ao modelo de advers�rio. Os algoritmos alvo
deste estudo s�o o SIGF, Clean-slate e INSENS, entendendo que no seu todo cobrem
todas tipologias de ataque em an�lise, no entanto cada um tem lacunas no seu
desenho quando direccionados alguns destes ataques. Ainda como contributo deste
trabalho, pretende-se modelar um ambiente de simula��o que permita
avaliar/analisar, no quadro de ataques definido, as caracteristicas dos
protocolos de encaminhamento em RSSF em m�teria de seguran�a e resist�ncia a
ataques. Portanto, importa estudar e estabelecer crit�rios para an�lise de
sistemas de simula��o para RSSF das as caracteristicas conhecidas, por exemplo
os recursos limitados dos modelos de energia, processamento e comunica��o.
% Palavras-chave do resumo em Portugu�s
\begin{keywords}
Redes de sensores sem fios,Protocolos de encaminhamento seguros,Simula��o
de redes de sensores, Ataque por intrus�o
\end{keywords}
% to add an extra black line
~\\ ~\\\rule{\textwidth}{0.2mm}




% Resumo em Ingls
% Se preferir, crie um ficheiro  parte e inclua com \include{}
\abstract
Sensor networks are an emerging technology in the field of monitoring, independently of physical
environments. They are formed by small devices that self-organize in order to cover a geographical
area can form a network of large scale with thousands of us. This autonomy and self-organization
presents some challenges related to security aspects, in particular, with respect to the routing of
data.

The work undertaken aims to contribute to the creation of a systemic model for the study of secure
routing protocols in sensor networks wireless (WSN). The definition of the type of player is the
initial step in the framework of different types of attack that was assessed. Coupled with the
formal model of Dolev-Yao, which focuses on the attacks on the media, the study of new models of
opponent-related intrusion and capture us is relevant and presented within the context of this work.
    In order to make the WSN resistant to some types of attacks outlined in this type of opponent,
have been developed several routing algorithms
insurance. The aim is to study some of these algorithms, representatives of the state of the art in
this field, establishing an array of measures of resistance to the type of opponent, which then
allows to evaluate the effectiveness of these.

    As a major contribution, this study highlight the design of an innovative simulation
environment, since they intend to implement features not found in simulation systems for the
existing WSN. It will provide the opportunity to design and evaluate routing algorithms are designed
to be safe when subject to attacks in the model defined adversary. This evaluation will focus
primarily on analysis of properties such as energy consumption, reliability, latency, accuracy of
data and correction of the behavior of the Protocol.

% Palavras-chave do resumo em Ingl�s
\begin{keywords}
Wireless Sensor Networks,Secure Routing Protocols,WSN Simulation,Intrusion
Atack
\end{keywords} 
% to add an extra black line
~\\ ~\\ \rule{\textwidth}{0.2mm}


% ndice. 
\tableofcontents

% Lista de figuras. Comente para ocultar
\listoffigures

% Lista de tabelas. Comente para ocultar
\listoftables

%% O contedo principal
\mainmatter

%  aconselhvel ter um captulo por ficheiro :
% "capitulo1.tex", "capitulo2.tex", ... "capituloN.tex" e depois
% inclu-los com:
\chapter{Introdu��o} \label{cap:introducao}
\onehalfspacing


\section{Introdu��o geral ou Motiva��o} \label{sect:introducao}
% 3 par�grafos por cada um do resumo
Os recentes avan�os no fabrico de \textit{hardware} de pequena dimens�o com
capacidades para desepenhar tarefas espec�ficas, tendo em conta tamb�m na
diminui��o do tamanho dos dispositivos de comunica��o sem fios, fez surgir, nos
ultimos anos um novo campo da investiga��o, as redes de sensores sem fios.
Uma rede de sensores � um conjunto de pequenos dispositivos de computa��o de
baixa pot�ncia destribu�dos num determinado ambiente com a fun��o de recolher
sensorialmente (ou monitorizar) fen�menos fisicos na ausencia de um observador.
Um ambiente pode ser um qualquer espa�o onde uma RSSF possa ser instalada por
exemplo: uma casa, um terreno de combate ou o pr�prio corpo humano. Entende-se
por fen�meno como sendo uma entidade ou caracteristica que � do interesse do
observador.

A componente b�sica e fundamental de uma rede de sensores � o n� sensor
(\textit{mote}) . Este pode conter diversos sensores para monitorizar, por
exemplo,  temperatura, luz ou movimento consoante as necessidades da aplica��o.
� caracterizado por ser pequeno, pouco poder de computa��o, baixa largura de
banda de comunica��o e energia limitada. Conhecidas estas limita��es, os
sensores t�m de ser distribu�dos em grande n�mero como forma de aumentar a
redund�ncia.




\section{Descri��o e contexto (ou descri��o do problema)} \label{sect:descricao}
A seguran�a nas RSSF � uma problema \textit{de
facto}, com impacto na sua aplica��o em sistemas cr�ticos. Apesar da evolu��o
do desenvolvimento de \textit{hardware}, dos n�s sensores, indicar que se
pretende alcan�ar dispositivos de muito baixo custo, com menos constragimentos
e mais capacidades, a investiga��o e desenvolvimento necessita de ferramentas
que permitam testar novos algoritmos, protocolos, topologias sem que se tenha
de instalar uma rede com sensores reais. A aplica��o real de sensores, na
investiga��o tem constragimento de ordem or�amental, que limita a dimens�o das
redes criadas e que n�o permite a repeti��o de experi�ncias , variando apenas
algumas condi��es predefinidas. Para suprir estas limita��es surge a
necessidade de criar sistemas de simula��o de RSSF. O que se tem vindo a
verificar � que, para cada problema cria-se uma ferramenta que permita
testar/avaliar o problema especifico. No entanto, existem determinados modelos
que s�o transversais a qualquer protocolo de encaminhamento que se queira
avaliar, por exemplo: consumo de energia, taxa de cobertura, fiabilidade,
escalabilidade. \\
No dom�nio do estudo da seguran�a em RSSF, n�o existe uma plataforma que
permita modelar protocolos de encaminhamento seguro enformados pelos ataques
mais comuns dirigidos a estas redes, de forma simples em que s� se esteja
centrado no desenho do protocolo. Aliada a esta necessidade, surge tamb�m a
necessidade de obter medi��es e resultados referentes a condi��es de
execu��o das redes, que permitam afinar os protocolos mediante os resultados
obtidos e a repeti��o de experi�ncias nas mesmas condi��es. 
\section{Solu��o apresentada (ou �mbito do trabalho)} \label{sect:solucao}
Assim, o desafio que se coloca � o desenho de um modelo de
avalia��o de protocolos de encaminhamento seguro em RSSF, que possibilite o
desenho e implementa��o destes protocolos com vista proporcionar mecanismos que
visem avaliar estes modelos. Por se tratar de quest�es de seguran�a, poder-se-�
ainda proporcionar o estudo do impacto dos ataques enformados pelo modelo
de advers�rio apresentado neste trabalho em cada um dos protocolos facilitando
assim a avalia��o poss�vel adequa��o dos protocolos implementados.
\section{Principais contribui��es previstas} \label{sect:contribuicoes}

 As principais contribui��es previstas devem poder ser descritas em n�o mais do
que uma p�gina, podendo adoptar-se, 
por exemplo, um estilo de apresenta��o por itens, com uma pequena descri��o de
um 
par�grafo associado a cada item.

 
\chapter{Trabalho relacionado} \label{cap:trabrelacionado}
Este cap�tulo apresenta uma vis�o do estado da arte relacionado com a seguran�a e modelos de
simula��o em RSSF. A primeira sec��o apresenta a defini��o do modelo de advers�rio e
tipologias de ataques. A segunda sec��o apresenta protocolos de encaminhamento seguro.  A
terceira sec��o apresenta diversos ambientes de simula��o relacionados com as RSSF e
\textit{ad-hoc}. Por fim, apresenta-se uma discuss�o e an�lise cr�tica do trabalho
relacionado, com vista a enquadra-lo com os objectivos da disserta��o a elaborar.
\section{Modelo de Advers�rio, Ataques ao Encaminhamento e
Contra-medidas} \label{sect:sec_mod_adversario_ataq_contramedidas}
\subsection{Arquitectura de Servi�os de Seguran�a em RSSF} \label{sect:subsec_arq_security_wsn}
Num sistema seguro � necess�rio que a seguran�a esteja integrada em cada um dos seus componentes, 
n�o se confinando a um componente isolado do sistema \cite{sec_in_wsn_perrig}. Nesta sec��o,
apresenta-se, introdutoriamente, alguns requisitos de seguran�a de uma RSSF e 
alguns servi�os de seguran�a, que representam um ponto de
partida para a garantia de propriedades de seguran�a, aquando do desenho de RSSF seguras.
\subsubsection{Requisitos de seguran�a de uma RSSF}
Os requisitos de seguran�a de uma RSSF podem variar consoante as especificidades da aplica��o que a
rede visa suportar. No entanto, apresentam-se, de forma gen�rica, os principais requisitos de
seguran�a de uma RSSF \cite{sec_in_wsn_perrig}:
\begin{descriptionNoIndent} 
\item[Autentica��o]
Sendo que o meio de comunica��o � partilhado, � necess�rio recorrer � autentica��o para garantir a
detec��o de mensagens alteradas ou injectadas no sistema, de forma n�o autorizada
\cite{sec_in_wsn_perrig}. O uso de criptografia assim�trica ainda n�o � vi�vel nas RSSF,
considerando as limita��es destas redes e as exig�ncias computacionais\footnote{N�o somente em
termos de mem�ria, mas tamb�m em termos de energia} destes mecanismos;
 \item[Confidencialidade]
Sendo uma RSSF uma infraestrutura baseada, fundamentalmente, na dissemina��o de dados recolhidos
sensorialmente em ambiente remoto e/ou n�o controlado e, normalmente, de f�cil acesso, � necess�rio
garantir a confidencialidade dos dados que circulam na rede. O uso de criptografia � o mais
indicado para este tipo de protec��o, sendo adequada a selec��o de algoritmos de encripta��o
fi�veis (ex: AES\footnote{\textit{Advanced Encryption System}} \cite{Stallings2005},
ECC\footnote{\textit{Elliptic Curve Cryptography}}\cite{Stallings2005}. Com a utiliza��o de chaves
criptogr�ficas, � necess�ria a adop��o de esquemas seguros de distribui��o de chaves
\cite{eschenauer2002}.
 \item[Disponibilidade]
Entende-se por disponibilidade a garantia do funcionamento de uma rede durante a totalidade do tempo
de opera��o. Os ataques que visam afectar esta propriedade s�o denominados por ataques de nega��o de servi�o (\textit{Denial of Service} - \textit{DoS}) \cite{Hu2005}. Para al�m de mecanismos que evitem estes ataques, � necess�rio garantir que a degrada��o da
rede (na presen�a de um ataque ) � controlada, ou seja, � proporcional ao n�mero de n�s
comprometidos;
 \item[Integridade]
A integridade garante que os dados recebidos por um n� n�o s�o alterados, por um
atacante, durante a transmiss�o. Em alguns casos, esta propriedade � garantida juntamente com a
autentica��o, usando mecanismos que permitem garantir ambas as propriedades numa s� opera��o. � comum o
uso de CMAC \footnote{\textit{Cipher based Message Authentication Code}} \cite{Stallings2005}, uma
vez que permite autenticar (com o uso de chave criptogr�fica sim�trica) e verificar a
integridade de uma mensagem \cite{SPINS}.
\item[Detec��o de Retransmiss�o Il�cita (ou Teste de Frescura da Mensagem)]
A frescura de uma mensagem garante que n�o � antiga
e/ou n�o foi reenviada por um atacante \cite{SPINS,Luk2007d}. Podem-se considerar dois
tipos de frescura: fraca (garantindo ordem parcial e sem informa��o do desvio de tempo,
usada para as medi��es dos sensores) e forte (que garante ordem total em cada
comunica��o, permitindo estimar o atraso, sendo usada para a sincroniza��o de tempo).
 \end{descriptionNoIndent}
\subsubsection{Servi�os B�sicos de Seguran�a}\label{sec:servicos_basicos_de_seguranca}
Alguns servi�os de seguran�a t�m vindo a ser desenvolvidos para as RSSF, com vista a garantir a
seguran�a ao n�vel da comunica��o (ex: criptografia, assinaturas, \textit{digests}). Estes servi�os
permitem que o arquitecto de sistemas se centre noutras problem�ticas relacionadas com o
comportamento dos protocolos face a ataques, por exemplo, de intrus�o. Apresentam-se de seguida
alguns servi�os mais comuns que representam as arquitecturas b�sicas de seguran�a para RSSF:
\begin{descriptionNoIndent}
\item[TinySec \cite{Karlof2004}]
TinySec � uma arquitectura para protec��o ao n�vel de liga��o de dados em RSSF. O objectivo
principal � o de providenciar um n�vel adequado de seguran�a, com o m�nimo consumo de recursos. Os
servi�os de seguran�a  disponibilizados s�o: autentica��o de dados (com a utiliza��o de
\textit{Message Authentication Codes}(MAC) \cite{Stallings2005}, em
particular o CBC-MAC\footnote{Cipher Block Chaining - Message Authentication Code (CBC-MAC))}) e
confidencialidade (CBC-MAC). N�o implementa nenhum mecanismo que garanta a frescura das mensagens,
tornando-o vulner�vel a ataques de retransmiss�o il�cita;
\item[MiniSec \cite{Luk2007d}]
Minisec � uma camada de rede concebida para possuir baixo consumo de energia (melhor que o
TinySec) e alta seguran�a. Uma das caracter�sticas principais, que a tornam mais eficiente, � o uso
do modo \textit{Offset Codebook} (OCB) \cite{Stallings2005} para encripta��o de blocos. Desta forma, � poss�vel, numa �nica passagem, autenticar e encriptar os dados, sem aumentar o tamanho da
mensagem, contribuindo para um menor consumo de energia. Esta arquitectura tem dois modos
de opera��o: um orientado para comunica��o \textit{unicast} (MINISEC-U) e outro para comunica��o 
\textit{broadcast} (MINISEC-B);
\item[SPINS \cite{SPINS}]
Conjunto de protocolos de seguran�a, constitu�do por dois componentes
principais: SNEP\footnote{Secure Network Encryption Protocol} \cite{SPINS} e
${\mu}$TESLA  \cite{SPINS,Luk2006}. O primeiro fornece servi�os de autentica��o e
confidencialidade \textit{unicast}, encriptando as mensagens (com
o modo CTR\footnote{\textit{Counter Mode}}) e protegendo-as com um MAC (CBC-MAC). O
SNEP gera diferentes chaves de encripta��o que derivam de uma chave-mestra, partilhada entre 
dois n�s, com um contador de mensagens para garantir a frescura de cada mensagem. O segundo componente, o
${\mu}$TESLA \cite{SPINS,Luk2006}, � um servi�o de autentica��o de \textit{broadcast}, que evita a
utiliza��o de mecanismos mais exigentes, de criptografia assim�trica, recorrendo a criptografia
sim�trica, autenticando as mensagens com um CMAC;
\item[Norma IEEE802.15.4 \cite{ietf_802154}]
Esta norma define a especifica��o da camada f�sica e de controlo de acesso ao meio das redes
pessoais de baixa pot�ncia (\textit{LRPAN}\footnote{Low Rate Personal Area Networks}). Foca-se,
essencialmente, na comunica��o entre dispositivos relativamente pr�ximos, sem a
necessidade de uma infraestrutura de suporte, explorando o m�nimo de consumo de energia. � uma norma
que j� se encontra implementada em algumas plataformas das RSSF (ex: Micaz \cite{micaz}).
Especifica alguns servi�os de seguran�a \cite{zigbee_802154},  representando uma primeira
linha de protec��o contra ataques � infraestrutura. Estes mecanismos s�o os seguintes: i) Cada
dispositivo mant�m uma lista de controlo de acessos (ACL) dos dispositivos confi�veis,
filtrando comunica��es n�o autorizadas; ii) Encripta��o de dados, partilha de uma chave
criptogr�fica entre os intervenientes na comunica��o; iii) Servi�o de integridade de cada
\textit{frame}, adicionando a cada \textit{frame} um \textit{Message Integrity Code}
(MIC) \cite{Stallings2005}; iv) Garantia de frescura de mensagens (\textit{Sequential Freshness}),
utilizando contadores de \textit{frames} e de chaves.
\item[ZigBee \cite{zigbee_802154,zigbee}]
Com a norma 802.15.4, orientada para as duas camadas mais baixas da pilha de
protocolos das RSSF, a norma ZigBee define as especifica��es para
a camada de rede e de aplica��o. J� incorpora alguns servi�os de seguran�a, nomeadamente: i)
Frescura, mantendo contadores associados a cada chave de sess�o, que s�o reiniciados em cada mudan�a
de chave; ii) Integridade, com op��es de integridade de mensagens que v�o desde os 0 aos 128 bits de
verifica��o; iii) Autentica��o, ao n�vel de rede e ao n�vel de liga��o de dados; iv)
Confidencialidade, com o algoritmo AES \cite{Stallings2005} com 128 bits.
Esta arquitectura utiliza o conceito de \textit{trusted center} para gest�o da seguran�a na rede,
implementando um coordenador de rede ZigBee. Este, acreditado por todos os n�s da rede, pode
desempenhar tr�s fun��es: i) Autentica��o de n�s participantes na rede; ii) Manuten��o e
distribui��o de chaves; iii) Seguran�a ponto-a-ponto entre n�s da rede.
\end{descriptionNoIndent}

\subsection{Modelo de Advers�rio} \label{sect:sec_mod_adversario_serv_seg}
A defini��o do modelo de advers�rio permite, desde logo, identificar as caracter�sticas e as 
capacidades dos atacantes e os ataques que estes podem desencadear na rede. Nesta sec��o,
caracteriza-se o modelo de advers�rio que enforma este trabalho.
\subsubsection{Modelo de Dolev-Yao}\label{sect:subsec_dolev_yao}
Um dos modelos de advers�rio mais conhecidos, quando se trata de an�lise formal de protocolos
seguros, � o modelo de Dolev-Yao \cite{Dolev1983}. Neste modelo, � considerado que a rede est�
sobre
o dom�nio do advers�rio o qual, perante este facto, pode extrair, reordenar, reenviar, alterar e
apagar
as mensagens que circulam entre quaisquer dois n�s leg�timos. Com esta assump��o, entende-se
portanto, que o advers�rio transporta a mensagem e, com isso, adopta um ataque do tipo 
\textit{man-in-the-middle} \cite{Stallings2005}, com comportamento incorrecto. Este funcionamento,
entenda-se, n�o � comparado � intrus�o mas sim � intercep��o de mensagens e pode ser mitigado pela
utiliza��o de mecanismos de criptografia.

As tipologias de ataque consideradas pelo modelo de advers�rio de Dolev-Yao
s�o  instanciadas pela norma X800 \cite{ITU-T1991}, que pretende
normalizar
uma arquitectura de seguran�a para o modelo OSI \cite{sd_tanenbaum}, atrav�s de uma abordagem
sistem�tica para o desenho de sistemas seguros. Esta norma considera a seguran�a
sob tr�s aspectos: ataque, mecanismo e servi�o de
seguran�a \cite{Stallings2005}. O
primeiro refere-se � forma usada para comprometer um sistema, por exemplo,
alterando ou tendo acesso n�o autorizado a dados desse sistema\footnote{ Na
literatura, algumas vezes usam-se os termos "ataque" e "amea�a" para denominarem o
mesmo efeito. No entanto, recorrendo ao RFC 2828 \cite{IETF2828}, podemos
definir amea�a como uma potencial viola��o de seguran�a, ou seja, � apenas uma
vulnerabilidade que pode ser explorada para desencadear um ataque. No caso do ataque, trata-se da
explora��o inteligente de uma ou
mais amea�as que resultam na viola��o, com sucesso, de um sistema que se pretendia
seguro}. O segundo aspecto considerado s�o os mecanismos de
seguran�a, que se entendem como o processo que permite detectar, prevenir ou
recuperar de um ataque � seguran�a (ex: encripta��o, controlo de acesso,
assinatura digital) \cite{Stallings2005}. Por fim, o terceiro aspecto define os
servi�os que, fazendo uso de um ou mais mecanismos de seguran�a, permitem resistir a
ataques dirigidos a determinada fonte de informa��o, quer seja durante o
processamento, quer seja durante a comunica��o. Considera-se, ent�o, que, para efeitos da futura
disserta��o,
os ataques subjacentes ao modelo de Dolev-Yao s�o protegidos a partir do estabelecimento de uma
camada b�sica de seguran�a, concretizada por uma das arquitecturas anteriormente referidas na Sec��o
\ref{sec:servicos_basicos_de_seguranca}. 
\subsubsection{Modelo de Intrus�o em RSSF}\label{sect:subsec_intrusao}
Considerando o estudo de seguran�a numa RSSF e, dada a sua exposi��o natural, nomeadamente a f�sica,
colocando cada sensor ao alcance de um advers�rio, torna-se relevante a considera��o de novos
modelos de advers�rio. Cada rede pode ser constitu�da por milhares de sensores e cada
um destes sensores � um ponto de poss�vel ataque \cite{Perriga}. Este ataque pode ser
tipificado como sendo por intrus�o ou captura. 

Este tipo de ataques pode ser desencadeado desde o n�vel
MAC \cite{Xiao2006} at� ao n�vel de intrus�o f�sica. Neste �ltimo, um actor externo
captura um ou mais sensores leg�timos e descobre os segredos criptogr�ficos. Este
facto permite-lhe replicar \cite{Parno2005} os segredos para sensores maliciosos, 
introduzido-os na rede de modo a que, agindo coordenadamente, possam comprometer a rede.
Conseguida a intrus�o, o atacante pode induzir, nos sensores leg�timos,
comportamentos incorrectos, baseados na informa��o falsa introduzida pelos
sensores maliciosos, influenciando o processo de encaminhamento. Estes ataques s�o de dif�cil
detec��o, uma vez que o car�cter aut�nomo das RSSF pode n�o permitir distinguir um comportamento
errado de uma falha. Com a intrus�o, um
sensor malicioso, embora respeitando o protocolo da rede, pode actuar de forma incorrecta,
levando a rede a criar topologias espec�ficas para determinado ataque ou for�ando toda a
informa��o a passar por n�s maliciosos, podendo estes suprimir ou violar a informa��o. 
\paragraph{Modelo bizantino: advers�rios bizantinos}
O modelo de ataques por intrus�o tem algumas parecen�as com as denominadas falhas
bizantinas \cite{falhas_bizantinas}, que s�o caracterizadas como 
falhas arbitr�rias com as quais um sistema n�o est�, � partida, preparado para lidar e que se
podem traduzir em comportamentos inesperados. Transpondo esta realidade para as
RSSF \cite{Fault_Intrusion_Tolerant_Techniques}, � dif�cil
detectar a introdu��o de n�s maliciosos, aut�nomos ou replicados a partir de um n� que foi	
comprometido. No entanto, alguns autores \cite{Parno2005,falhas_bizantinas} t�m-se
debru�ado sobre esta problem�tica, a fim de dotarem os
algoritmos de encaminhamento com mecanismos que permitam detectar a replica��o de n�s maliciosos
numa RSSF.
Para se lidar com ataques com comportamentos bizantinos, recorre-se a mecanismos
probabil�sticos que, ainda que possam n�o mitigar o ataque por completo, aumentam a resili�ncia e
acabam por transformar um ataque num mal menor, definindo at� onde pode ser comprometida a rede, por
forma a, ainda assim, garantir a fiabilidade necess�ria para o seu funcionamento.



 

\subsection{Ataques ao Encaminhamento} \label{ataques_encaminhamento}
Apesar de existirem ataques que podem ser dirigidos a qualquer uma das camadas da pilha da RSSF, nesta
sec��o apresentam-se os ataques relacionados com a camada de rede, respons�vel pelo
encaminhamento de dados. Os protocolos de encaminhamento em MANETs \cite{Corson1999} e em RSSF, de
uma forma geral, decomp�e-se em tr�s fases: descoberta dos caminhos, selec��o dos caminhos e
manuten��o da comunica��o pelos caminhos seleccionados. Os ataques a um algoritmo de encaminhamento,
normalmente, podem explorar as vulnerabilidades de cada uma destas fases de forma espec�fica. Em
seguida, os ataques s�o associados � fase do protocolo em que se podem desencadear e s�o
apresentadas, tamb�m, as contra-medidas que permitem mitig�-los.
\subsection{Ataques � organiza��o da rede e descoberta de n�s} \label{sect:subsec_ataq_org_rede}
Nos protocolos do tipo \textit{table-driven} \cite{al-karaki_routing_2004}, ap�s a descoberta dos
n�s vizinhos � necess�rio recolher informa��o para a constru��o das tabelas
de encaminhamento. No entanto, em protocolos do tipo \textit{on-demand}
\cite{al-karaki_routing_2004}, esta fase � desencadeada em cada in�cio de transmiss�o. Este
funcionamento corresponde � organiza��o e descoberta de n�s numa RSSF.
\begin{descriptionNoIndent}
\item[Falsifica��o de Informa��o de Encaminhamento]
Este ataque tem impacte na forma��o da rede e na descoberta dos n�s. Induz a cria��o de entradas
incorrectas nas tabelas de encaminhamento, podendo tamb�m fazer com que estas fiquem lotadas e
inv�lidas. Nos protocolos \textit{on-demand}, o impacte pode ser menor, uma vez que obriga o 
atacante a injectar informa��o errada a cada ciclo de transmiss�o. Outro ataque que causa
estes mesmos efeitos � realizado por n�s atacantes que inundam a rede com pacotes do tipo
\textit{Route Request} (RREQ), pondo em causa a disponibilidade da rede.
\item[Ataques \textit{Rushing}]
O \textit{Rushing attack} \cite{Rushing_attacks_perrig} � definido pela explora��o, por parte do
atacante, de uma janela de tempo para responder a um pedido de caminho para um destino. Este ataque
� efectivo quando um protocolo (ex:
AODV \cite{Perkins1999}) aceita a primeira resposta que recebe \textit{Route
Reply} (RREP).
Explorando isto, o atacante � sempre um candidato a ser o pr�ximo encaminhador, uma vez que n�o
respeita temporizadores nem condi��es de resposta, podendo depois influenciar o estabelecimento das
rotas.
\end{descriptionNoIndent}
\subsubsection{Contra-medidas}
Os mecanismos de autentica��o fazem com que ataques de falsifica��o de informa��o ou de inunda��o
de RREQ sejam minimizados. Os n�s da rede podem partilhar chaves sim�tricas (par-a-par) como forma
de autenticar as mensagens de dados e controlo do encaminhamento (RREQ e RREP). Desta forma, o
atacante, n�o possuindo as chaves necess�rias para a comunica��o, n�o poder�
participar no protocolo.

Para fazer face a ataques de \textit{Rushing}, alguns autores \cite{Rushing_attacks_perrig}
apresentam dois mecanismos de defesa: reenvio aleat�rio de RREQ
(\textit{Randomized RREQ Forwarding}) e detec��o segura (\textit{Secure
Detection}). No primeiro caso, cada n�
interm�dio guarda  um conjunto de mensagens RREQ, escolhendo depois,
aleatoriamente, um para reenviar.
Ainda assim, pode ser seleccionada uma mensagem RREQ maliciosa, da� a exist�ncia do segundo
mecanismo, que permite a autentica��o de mensagens entre dois n�s, garantindo que estas pertencem a
n�s leg�timos. Outros mecanismos passam pela selec��o de mais do que uma resposta (permitindo que a
mensagem seja enviada por outro caminho) ou pela colec��o de v�rias respostas (escolhendo,
aleatoriamente, uma para responder).
\subsection{Ataques ao estabelecimento de rotas} \label{sect:subsec_ataq_est_rotas}
Os ataques desencadeados nesta fase aumentam a probabilidade de um atacante pertencer a uma rota.
Estabelecida a rota atrav�s de si pode alterar as mensagens ou agir de forma a desencadear ataques
na fase de manuten��o de rotas.
\begin{descriptionNoIndent}
\item[\textit{HELLO Flooding}]
Este ataque explora os protocolos que se anunciam aos vizinhos, emitindo
mensagens de \textit{HELLO} \cite{Survey_wsn_Sec_issues,Karlof2003}.
Os protocolos baseados na localiza��o podem ser vulner�veis a este ataque, uma vez que, com um
dispositivo do tipo \textit{laptop-class} \cite{Karlof2003}, que possua um alcance r�dio
suficientemente potente para cobrir toda a rede, � poss�vel anunciar-se a todos os n�s como vizinho,
for�ando a informa��o a fluir atrav�s dele.
\item[Ataque \textit{Sinkhole}]
No ataque \textit{sinkhole} \cite{Sinkhole_attack}, o atacante, induz os n�s da rede a fazerem
passar a informa��o por dele. Assim, anuncia-se aos n�s vizinhos,  como tendo boa comunica��o com o
n� \textit{sink}, tornando-se, assim, um ponto de passagem da informa��o. O ataque � realizado
enviando pacotes de RREQ, alterando a origem e aumentando o n�mero de sequ�ncia, como forma de
garantir que esta informa��o se sobrep�e a qualquer informa��o leg�tima. Assim,
um atacante poder� participar num n�mero elevado de rotas, podendo alterar ou encaminhar,
de forma selectiva,  a informa��o. Os protocolos \textit{table-driven} s�o vulner�veis a estes
ataques, enquanto os protocolos baseados em localiza��o n�o o s�o, no caso das suas rotas serem
estabelecidas \textit{on-demand} \cite{Karlof2003,Survey_wsn_Sec_issues,Attaks_defenses_sec_in_wsn}.
\item[Ataque \textit{Wormhole}]
Neste tipo de ataque, apresentado por Perrig \textit{et al} \cite{Wormhole_perrig}, dois n�s
maliciosos colaboram para a realiza��o do ataque. Os atacantes estabelecem uma liga��o (em geral, de
melhor qualidade) para comunicarem entre si, permitindo a um n� malicioso capturar pacotes ou partes
de pacotes e envi�-los pela liga��o privada para o outro atacante, noutro extremo da rede. Este
ataque � particularmente eficaz em
redes RSSF baseadas em localiza��o que, caso sejam comprometidas, n�o conseguir�o
estabelecer caminhos maiores do que dois \textit{hops}
\cite{perrig_survey_ad_hoc,Survey_wsn_Sec_issues}.
Para al�m disso, os atacantes transformam-se em n�s muito solicitados, pois
apresentam-se aos outros n�s como tendo uma melhor liga��o e estando a menor
dist�ncia do destino.
\item[Ataque \textit{Sybil}]
Este ataque foi definido como uma ac��o que permitia atingir os mecanismos de redund�ncia
em armazenamento distribu�do em sistemas ponto-a-ponto \cite{Douceur2002}. Outra
defini��o que surge, agora associada �s RSSF, � a que o define como ``um dispositivo malicioso que
ilegitimamente assume m�ltiplas entidades'' \cite{sybil_perrig}. Com estas defini��es e, devido � sua
taxonomia, � um ataque bastante efectivo contra protocolos de encaminhamento
\cite{Karlof2003}, em particular, nos protocolos que adoptam m�ltiplos caminhos,
o que permite que um n� assuma
m�ltiplas identidades, ocultando o facto de os dados estarem a passar por um
�nico n� malicioso
\cite{Survey_wsn_Sec_issues,Attaks_defenses_sec_in_wsn}.
\end{descriptionNoIndent}
\subsubsection{Contra-medidas}
Uma das formas de prevenir um ataque HELLO \textit{flooding }\cite{Karlof2003} � a implementa��o de
mecanismos de resposta (\textit{acknowlege}) a an�ncios HELLO. Desta forma, caso o meio de
comunica��o do atacante cubra toda a rede, um n� leg�timo, que n�o o alcance e, portanto, n�o receba
a  resposta, n�o considerar� o an�ncio como v�lido.  � poss�vel  proceder � autentica��o da
mensagem, certificando-a numa entidade central que, ao detectar que um n� se anuncia como vizinho de
muitos outros n�s, toma precau��es, repudiando o n� perante a rede \cite{Survey_wsn_Sec_issues}.

Alguns autores t�m vindo a desenvolver algoritmos que visam a detec��o de ataques do tipo
\textit{Sinkhole} \cite{Sinkhole_attack}. Um desses mecanismos � o \textit{Sinkhole
Intrusion Detection System} (SIDS) \cite{Sinkhole_attack}, orientado para a detec��o destes ataques
ao protocolo DSR \cite{DSR}. Este sistema prop�e tr�s mecanismos de detec��o: i)
Descontinuidade de n�meros de sequ�ncia (tendo em conta que um atacante tentar� usar n�meros de
sequ�ncia muito grandes, um n� pode identificar os que crescem rapidamente ou os que n�o
respeitam uma ordem crescente, considerando-os um ataque); ii) Verifica��o de pacotes (os
vizinhos podem certificar a origem dos pacotes enviados por um n�. Isto ser� dif�cil de
realizar nos pacotes atacantes, uma vez que a origem � alterada. Assim, se circularem muitos pacotes
n�o certificados poder-se-� detectar que a rede est� sob ataque); iii) N�mero de caminhos
a passar por um n� (cada n� pode observar a sua tabela de encaminhamento e detectar que existem
muitos caminhos a passar pelo mesmo n�, logo pode estar na presen�a de um ataque
\textit{Sinkhole} \cite{Survey_wsn_Sec_issues,Attaks_defenses_sec_in_wsn}).
A utiliza��o de chaves ponto-a-ponto, como forma de garantir que a informa��o dos pacotes
� leg�tima, evita que um atacante altere dados da mensagem (origem e n�mero de
sequ�ncia) \cite{SIGF,INSENS}.

A utiliza��o de \textit{packet leashes} \cite{packet_leashes_perrig} permite mitigar o ataque
\textit{wormhole}. Assim, existem dois tipos de condi��es para se aceitar os pacotes vindos de uma
origem: a localiza��o e o tempo. A primeira permite que um n� receptor, conhecendo a
localiza��o da origem, saiba se um pacote atravessou a rede por um \textit{wormhole}, calculando
a dist�ncia entre os dois pontos. O segundo baseia-se, essencialmente, no tempo de transmiss�o
do pacote, exigindo, ent�o, a sincroniza��o de rel�gios. Se for muito r�pido a chegar ao destino, este
n� assume que est� perante um ataque de \textit{wormhole}. A implementa��o de encaminhamento por
m�ltiplas rotas, tamb�m, permite fazer face a ataques \textit{wormhole} \cite{clean_slate}.

Para o ataque \textit{sybil}, s�o
poss�veis dois esquemas de protec��o \cite{sybil_perrig,Survey_wsn_Sec_issues}:
i) \textit{Radio resource testing} (cada vizinho s� pode transmitir num
canal, seleccionando um canal para receber e enviar uma mensagem. Os n�s que n�o
responderem s�o
tratados como falsos, ao n�vel MAC); ii) \textit{Random key distribution} (usando um modelo de
\textit{key-pool}, s�o atribuidas $n$ chaves de um conjunto de $m$. Se dois n�s partilharem $q$
chaves ent�o estar�o em condi��es de comunicar de forma segura. Uma fun��o de dispers�o, com base no
identificador do n�, permite gerar chaves, evitando que um n� possa conhecer uma grande parte das
chaves). A no��o de reputa��o dos n�s vizinhos pode tamb�m permitir detectar comportamentos
incorrectos de atacantes \textit{sybil} \cite{SIGF} ou, alternativamente, realizar o an�ncio dos
vizinhos de forma autenticada \cite{clean_slate}.
\subsection{Ataques � manuten��o de rotas} \label{sect:subsec_ataq_manut_rotas}
\begin{descriptionNoIndent}
\item[Ataque \textit{Blackhole}]
No ataque \textit{blackhole} \cite{HongmeiDeng2002}, o atacante intercepta os pacotes destinados ao
n�/�rea alvo de ataque, informando a origem de que se trata de um caminho de melhor qualidade ou a
menor dist�ncia, for�ando todo o tr�fego, dirigido ao destino, a circular
atrav�s dele. Assim, um
n� malicioso interm�dio, pode anunciar-se com um caminho melhor, apesar de n�o ter sequer caminho
para o destino, originando um ``vazio'' e interrompendo o processo de comunica��o
\cite{Survey_wsn_Sec_issues,Attaks_defenses_sec_in_wsn}.
\end{descriptionNoIndent}
\subsubsection{Contra-medidas}
Para mitigar os ataques de \textit{blackhole} existem v�rias propostas
 \cite{blackhole_adhoc,Attaks_defenses_sec_in_wsn,HongmeiDeng2002}, das quais se destacam as
que implementam os seguintes mecanismos: i) Confirma��o do destino, em que � enviada uma mensagem
de ACK por cada mensagem recebida, pelo caminho inverso; ii) Defini��o de
limites de tempo de recep��o das mensagens de ACK ou de mensagens de
falha; iii) Mensagens de falha, geradas quando um n� interm�dio detecta o fim do
temporizador de ACK; iv) Caminho definido pela origem, significando que, em cada pacote, � indicado,
pela origem, o caminho seguido at� ao destino. Os mecanismos que n�o se baseiem em informa��o
qualitativa do caminho tamb�m permitem resistir a estes ataques \cite{clean_slate}.

\subsection{Ataques por Intrus�o/Replica��o}
Alguns dos ataques tipificados anteriormente podem ser desencadeados a partir de n�s
maliciosos \cite{Sinkhole_attack} introduzidos na rede de forma incorrecta e posteriormente
replicados.  Os ataques por intrus�o nas RSSF s�o tipificados pela capacidade de um
atacante se apropriar de material criptogr�fico que, durante o desenrolar do protocolo de
encaminhameanto, lhe permita participar nas comunica��es passando por um n� leg�timo. Este
ataque quando conseguido � bastante devastador para a rede, como por exemplo, quando utilizado para
coordenar um ataque \textit{sybil}. 

Os ataques por replica��o \cite{Parno2005} correspondem � introdu��o de novos n�s da rede clonados
de n�s leg�timos mas que possuem comportamentos incorrectos, como seja a transmiss�o de informa��o
para outros n�s atacantes, originando ataques tipificados na Sec��o \ref{ataques_encaminhamento}.
Estes ataques, resultantes da replica��o, s�o particularmente efectivos em sistemas 
do tipo de vota��o, ou cuja a opera��o da rede dependa de mecanismos de elei��o. Pode-se ent�o
dizer que, mitigar os ataques por intrus�o permite que, � partida, se possam reduzir algumas
condi��es para a indu��o de outros ataques.
\subsubsection{Contra-medidas}
Apresentam-se algumas contra-medidas para ataques por intrus�o, algumas das quais t�m sido
incorporadas em protocolos de encaminhamento:
\begin{descriptionNoIndent}
\item[Autentica��o central] Uma primeira forma de defesa contra replica��o � a
autentica��o dos n�s, usando os seus identificadores, realizada numa esta��o central, permitindo
detectar inconsist�ncias ou duplica��es.
\item[M�ltiplas Rotas] A exist�ncia de diversos caminhos para entregar uma mensagem
no destino, faz com que se possa aliar a toler�ncia a falhas � resist�ncia a intrus�es. Assim, se a
mensagem for interceptada por um intrusor, alta probabilidade desta alcan�ar o
destino usando outro dos caminhos de envio \cite{INSENS}.
\item[Detec��o de replica��o] Perrig \textit{et} Parno em \cite{Parno2005}
apresentam
mecanismos de detec��o distribu�da de replica��es. Um dos mecanismos � de car�cter aleat�rio
\textit{Randomized Multicast} e outro, denominado por \textit{Line-Selected Multicast}, que se
serve do modelo de comunica��o multi-\textit{hop} da rede para detec��o de n�s duplicados. Outro
exemplo, � a cria��o de estruturas auxiliares (ex: �rvores bin�rias) como acontece no protocolo 
\textit{Clean-Slate} \cite{clean_slate}. 
\end{descriptionNoIndent}


\newpage
\section{ Estudo de Protocolos de Encaminhamento Seguro para RSSF}
Como ponto introdut�rio da discuss�o e apresenta��o de algoritmos de encaminhamento em RSSF,
importa identificar algumas tipologias ou classes destes algoritmos.
\subsection{Caracteriza��o dos protocolos de encaminhamento em RSSF}
Podem-se estabelecer tr�s classes de protocolos\cite{Akkaya2005}: os baseados na localiza��o,
os centrados nos dados e os hier�rquicos. Os protocolos baseados na localiza��o usam esta informa��o
para tomarem as melhores decis�es para alcan�ar os destinos(ex: IGF\cite{igf_protocol}). Os
centrados nos dados, ou seja, os que exploram a redund�ncia e a sem�ntica dos dados, normalmente s�o
baseados em algoritmos que efectuam pesquisas lan�adas a partir de n�s de sincroniza��o (ex:
Directed Diffusion\cite{DirectDifusion}). Por fim, os protocolos hier�rquicos, cuja concep��o �
baseada na constru��o de grupos de n�s, normalmente definidos como \textit{clusters}(ex:
LEACH\cite{leachprotocol}), que funcionam no principio de agrega��o de dados do grupo e
transfer�ncia da informa��o para os n�s base.

Para al�m destas classifica��es podemos ainda considerar algoritmos quanto ao momento em que s�o
determinadas as  rotas de encaminhamento de dados\cite{al-karaki_routing_2004}. Assim, consideram-se
os protocolos como \textit{table-driven} ou \textit{on-demand}.  Os primeiros referem-se a
protocolos que mant�m as tabelas de encaminhamento trocando mensagens de controlo durante a sua
opera��o. Assim, observa-se um maior consumo de energia devido � regular troca de mensagens. No
segundo caso, nos protocolos \textit{on-demand}, as rotas s�o determinadas em cada envio de
mensagem. Apesar de representar alguma sobrecarga, em cada envio, acaba por compensar em redes mais
m�veis e com eventos mais espa�ados. 

\subsection{Protocolos de encaminhamento seguro em RSSF}

Muitos dos protocolos de encaminhamento para RSSF n�o foram desde logo concebidos tendo em conta
o factor da seguran�a\cite{Karlof2003,al-karaki_routing_2004}, antes, pretendiam adaptar-se ao
ambiente das aplica��es e �s caracter�sticas e capacidades das RSSF. No entanto, quando se pretende
estender a sua utiliza��o para outros dom�nios, cuja seguran�a � indispens�vel, estas preocupa��es
aumentam, uma vez que os mecanismos de seguran�a implicam directamente um aumento da computa��o e
pode implicar um aumento no custo da comunica��o, reflectindo-se na autonomia dos sensores.

Nesta sec��o apresentam-se alguns protocolos de encaminhamento seguro em RSSF. Visam
cubrir todo o espectro da tem�tica deste trabalho e apresentam no seu todo os mecanismos de
seguran�a que se pretende estudar. 

\subsubsection{\textit{Secure Implicit Geographic Forwarding }(SIGF)}
Conhecida a inexist�ncia de mecanismos de seguran�a em alguns dos algoritmos de encaminhamento de
RSSF, a implementa��o destes mecanismos correspondeu a um primeiro passo neste dom�nio. Um destes
casos foi o algoritmo de encaminhamento \textit{Implicit Geographic Forwarding} 
(IGF)\cite{igf_protocol}, que deu origem a uma implementa��o segura o
SIGF\cite{SIGF}. 

O IGF � um protocolo \textit{on-demand}, baseado na localiza��o, que n�o mantendo o estado ao longo
do seu funcionamento, faz com que funcione sem que seja necess�rio o conhecimento da topologia da
rede ou a presen�a de outros n�s. O seu car�cter n�o determin�stico de encaminhamento, j� representa
um mecanismo de seguran�a perante determinados ataques, mas, n�o � de forma alguma suficiente para
manter uma aplica��o, com requisitos de seguran�a, a executar em ambientes cr�ticos.
\paragraph*{\textbf{Funcionamento do protocolo IGF}}
No protocolo IGF o ambiente est� definido por coordenadas que permitem a cada n� saber exactamente
a sua localiza��o. Com a agrega��o do n�vel de rede com o n�vel 
MAC\footnote{ \textit{Medium Access Control}} num �nico protocolo \textit{Network/MAC} �
poss�vel\cite{igf_protocol}, no momento de envio do pacote, determinar qual o melhor pr�ximo
candidato por onde encaminhar os dados. O protocolo inicia com a origem a enviar uma mensagem do
tipo \textit{Open Request To Send} (ORTS) para a vizinhan�a (com a localiza��o e o destino). Cada n�
que se encontre no sextante v�lido \footnote{�ngulo de 60� centrado na origem orientado para o
destino e determinado por cada n� com base na sua localiza��o} inicia um temporizador de CTS
(\textit{Clear To Send}) inversamente proporcional a determinados par�metros (dist�ncia � origem,
energia existente e dist�ncia perpendicular ao destino), favorecendo os n�s com melhores condi��es.
Ao expirar o temporizador, � enviada uma mensagem de CTS que, ao ser recebida possibilita o in�cio
de mensagens do tipo DATA apartir da origem. Como este protocolo n�o mant�m estado,resiste a
mudan�as de topologia da rede, o facto de escolher em cada envio o n� seguinte constitui um
mecanismo de toler�ncia a falhas e que, em caso de ataque, confina os danos, � vizinhan�a do n�
comprometido.
\paragraph*{\textbf{Funcionamento do protocolo SIGF\cite{SIGF}}}
A introdu��o de mecanismos de seguran�a, num protocolo existente, compreende um acr�scimo de
sobrecarga no funcionamento do protocolo. Contudo, o protocolo SIGF\cite{SIGF} pretende
manter um bom desempenho e uma elevada taxa de sucesso de entrega das mensagens, mesmo durante um
ataque. Uma das caracter�sticas deste protocolo � o facto de ser configur�vel e, como tal, permitir 
adaptar os mecanismos de seguran�a ao grau de amea�a. O protocolo tem
tr�s extens�es ao protocolo IGF\cite{igf_protocol} que possibilita a evolu��o gradual de um
protocolo seguro sem estado para um protocolo com manuten��o de estado, e com isto
mais pesado e exigente em recursos.

A primeira extens�o � a mais simples e a menos exigente em recursos, o SIGF-0. Continua a n�o manter
o estado e a ter um car�cter n�o determin�stico. No entanto, n�o sucumbe a ataques do tipo
\textit{rushing}\cite{Rushing_attacks_perrig} por n�o emitir logo para ao primeiro n� que envie o
CTS mas sim manter um conjunto de poss�veis candidatos a pr�ximo n�. A extens�o interm�dia, SIGF-1
j� mant�m estado, mas ao n�vel local, podendo com isto constituir listas de reputa��o dos seus
vizinhos por forma a escolher melhor pr�ximo n�. Por fim, e tratando-se j� de um protocolo mais
robusto, mas mais exigente,  o SIGF-2 partilha o estado com os seus vizinhos. Permitindo usar
mecanismos criptogr�ficos que permite garantir integridade, autenticidade, confidencialidade e
frescura. Ainda assim, acumula as propriedades de seguran�a de cada um dos seus protocolos
antecessores SIGF-0 e SIGF-1.
% \end{description}
\subsubsection{\textit{ INtrusion-tolerant routing protocol for wireless
SEnsor NetworkS }(INSENS)}
Este protocolo \cite{INSENS} foi concebido tendo em vista a toler�ncia a intrus�es e como
tal faz face a uma das tipologias do modelo de advers�rio preconizado neste trabalho. Para cumprir
com este objectivo, foram identificados dois tipos de ataques: ataques por
nega��o de servi�o\cite{Hu2005} e ataques ao encaminhamento. O protocolo assenta na exist�ncia de
uma esta��o base, constituindo-se como um centro confi�vel, que partilha chaves criptogr�ficas
sim�tricas com cada um dos n�s da rede. Este caracter�stica permite que, em caso comprometimento de
um n�, o atacante n�o ter� acesso a mais do que uma chave segura da rede,  isolando, de alguma
forma, o ataque.

O uso de caminhos redundantes permite aumentar a resili�ncia a atacantes n�o detectados.
Bastando que exista apenas um caminho sem interposi��o de atacantes, para que as mensagens cheguem
ao destino sem serem comprometidas. Note-se que neste protocolo n�o � poss�vel a comunica��o
directa entre n�s da rede, sem que esta n�o passe pela esta��o base. 

O papel fundamental do protocolo, em termos de encaminhamento seguro, � desempenhado pela esta��o
base. Uma das vantagens apontadas pelos autores � a redu��o das computa��es nos n�s da
rede (ex: para gera��o de chaves, tabelas de encaminhamento), cuja limita��es s�o as conhecidas.
Assim, a forma��o das tabelas de encaminhamento divide-se em tr�s fases: Pedido de rotas
(\textit{route request}); Recolha dos dados de encaminhamento; Propaga��o das rotas.
A primeira fase, corresponde ao envio por parte da esta��o base de uma mensagen destinada a todos
os n�s da rede por forma a obter dados sobre as vizinhan�as. Numa segunda fase, cada n�, responde
com a sua vizinhan�a para esta��o base. Por fim, depois da esta��o base tratar toda a informa��o
recolhida s�o elaboradas as tabelas de encaminhamento, que s�o depois propagadas para cada n�.
Podendo prosseguir-se com o encaminhamento dos dados baseando nas tabelas recebidas.

\subsubsection{\textit{Secure Sensor Network Routing: Clean-Slate approach}}
O algoritmo Clean-Slate\cite{clean_slate} foi concebido para uso generalizado, desenhado desde de
inicio, de forma sistem�tica, com caracter�sticas de seguran�a. � orientado para a
comunica��o ponto-a-ponto entre n�s da rede, visando a resist�ncia mesmo na presen�a de um
ataque (ataque activo). Classifica-se como um protocolo \textit{table-driven}.
\paragraph*{\textbf{Funcionamento do Protocolo}}
Cada sensor da rede recebe um identificador �nico global, um certificado assinado por uma
autoridade de certifica��o da rede (AR), a chave publica desta entidade e um conjunto de valores
(desafios) baseados numa fun��o de dispers�o de um sentido (\textit{one way hash function}).
Neste protocolo, podem-se identificar tr�s fases de opera��o: organiza��o da rede,
estabelecimento dos caminhos  e manuten��o das rotas.

O protocolo estabelece as tabelas de encaminhamento e os endere�os din�micos (de tamanho
vari�vel) para cada n� da rede, usando um algoritmo recursivo de agrupamento, que executa 
de forma determin�stica mediante uma topologia. Os grupos s�o formados de forma recursiva e
hier�rquica fundindo-se at� que a rede forme apenas um �nico grupo. Em cada fus�o �
acrescentado � esquerda um bit (0/1) que permitir� distinguir o endere�o de cada n�. Dentro de um
mesmo grupo a comunica��o � feita usando \textit{broadcast} autenticado inspirado nos protocolo
${\mu}$TESLA\cite{SPINS,Luk2006}.

Este algoritmo incorpora mecanismos de detec��o de comportamentos incorrectos dos n�s, por exemplo,
caso pretendam assumir m�ltiplas identidades(\textit{sybil}\cite{sybil_perrig,Douceur2002}). Este
mecanismo � desencadeado ap�s a forma��o dos grupos, com cada n� a anunciar o seu endere�o para os
vizinhos, aplicando-se um algortimo de detec��o de replica��o de n�s\cite{Parno2005}. Outro
mecanismo para a detec��o de forma��o incorrecta de grupos � a utiliza��o de \textit{Grouping
Verification Trees (GVT)}, baseado em tabelas de dispers�o que providenciam autentica��o ao n�vel
das folhas usando a raiz para certifica��o. Cada n� tem uma GVT permitindo verificar qualquer
comunica��o trocada com outros n�s da rede.

Durante a fase de manuten��o das rotas e encaminhamento, o algoritmo incorpora opera��es que
permitem tratar a saida e entrada de n�s. Um n� ao detectar a sa�da de outro,
procura num dos vizinhos um novo n� fronteira que lhe permita alcan�ar o mesmo grupo antes
acessivel pelo n� ausente. A defini��o de �pocas (\textit{ephocs}), permite que ao fim de algum
temaopo o algoritmo de agrupamento se repita por forma a incluir novos n�s. No que respeita ao
encaminhamento usa m�ltiplas rotas, fazendo com que o protocolo possa contornar �reas comprometidas
da rede. Os n�s maliciosos s�o retirados do algoritmo usando uma t�cnica denominada por
\textit{Honeybee}, que corresponde a: quando um n� malicioso (pode ser replicado) � detectado, a
rede � inundada com um pacote que indica que o atacante deve ser retirado das tabelas e, tratando-se
de uma replica��o, o n� replicado deve-se sacrificar.

De forma sum�ria, este protocolo incorpora os tr�s conceitos para o desenho de protocolos de
encaminhamento seguro: preven��o (autentica��o), resili�ncia (multiplas rotas) e
detec��o/recupera��o (GVT/Honeybee). Implementando-os todos, ao contr�rio do que acontece com alguns
protocolos que apenas implementam um destes conceitos. Transformando-o num protocolo base, indicado
para o estudo comparativo com outros protocolos.
\newpage
\section{Ambientes de Simula��o}
Os ambientes de simula��o de RSSF surgem como uma necessidade, inevit�vel, para o
teste e desenvolvimento das redes de sensores e de todas as tecnologias
associadas \cite{park_simulating_2001,simulators}. Alguns ambientes t�m sido desenvolvidos
especificamente para determinados problemas. Outros s�o adaptados a partir de ambientes j�
existentes, como � o caso do NS2 \cite{ns2} ou J-Sim \cite{jsim}, que foram concebidos para
simula��es relacionadas com redes convencionais (ex: IEEE802.3, IEEE802.11). A caracter�stica
importante destes ambientes � a capacidade de repeti��o de
experi�ncias, perante as mesmas condi��es, facilitando, assim, uma an�lise sistem�tica do objecto de
estudo.

Nesta sec��o, apresentam-se diversos ambientes de simula��o, mais comuns, e que permitam simular um
sistema de RSSF. Foram seleccionados, em primeiro lugar, seguindo crit�rios relacionados com 
engenharia de \textit{software}. Seguidamente, com vista � avalia��o de um ambiente que se mostre
adequado para ser utilizado no trabalho de disserta��o, foram estabelecidos crit�rios relacionados
com as RSSF.

\subsection{Crit�rios Relacionados com Engenharia de Software}
\begin{description}\descvspace	
 \item[Portabilidade da Linguagem] Devido �s caracter�sticas da linguagem de programa��o Java,
inerente ao seu ambiente de execu��o, � consequente portabilidade e � programa��o orientada a
objectos, foram seleccionados apenas ambientes desenvolvidos nesta linguagem. 
\item[C�digo Aberto e
Livre] Esta propriedade permite que se contornem
obst�culos inerentes a licenciamento de \textit{software}, ao mesmo tempo que 
possibilita a an�lise e aproveitamento de todas as funcionalidades existentes,
permitindo introduzir algumas melhorias ou altera��es espec�ficas.
\item[Modularidade e extensibilidade] Tendo em conta que os ambientes n�o
possuem todos as mesmas caracter�sticas e funcionalidades e, considerando que
a componente experimental da disserta��o ir� introduzir novos
mecanismos, o princ�pio da modularidade e da f�cil extensibilidade facilitar� o desenrolar do
trabalho.
\item[Documenta��o] Alguns ambientes podem n�o estar bem documentados. Este crit�rio ser�
importante como ponto de partida para o aprofundamento do conhecimento de cada uma das
arquitecturas destas ferramentas.
\end{description}
\subsection{Crit�rios Relacionados com as RSSF}
\begin{description}\descvspace
\item[Escalabilidade da Rede] Uma das caracter�sticas mais importantes das RSSF � o conceito de
escala, que se deve ao facto de estas compreenderem, normalmente, um grande n�mero de sensores
distribu�dos por uma vasta �rea. Assim, � importante que o ambiente de simula��o possa suportar
experi�ncias com milhares de n�s, uma vez que o factor escala � um das propriedades que se querem
ver analisadas no trabalho a desenvolver;
\item[Modelo de Colis�es/Comunica��o R�dio] � fundamental que este modelo se encontre presente no
sistema de simula��o, por ser um componente base das RSSF, relacionado com a camada de
acesso ao meio e de liga��o de dados (ex: B-MAC, S-MAC) \cite{BMAC,SMAC};
\item[Modelo de Gest�o de Energia] A exist�ncia de um modelo de energia permitir� adaptar esta
funcionalidade e inclu�-la na plataforma final, visto tratar-se de uma das propriedades
que se deseja estudar.
\item[Capacidade de Emula��o] Alguns simuladores possuem a capacidade de emular 
um sensor real, permitindo efectuar o carregamento de c�digo directamente para o \textit{mote}, sem
recurso a recompila��o. N�o sendo um crit�rio mandat�rio, reveste-se de algum interesse;
\item[Modelo de Mobilidade] Ainda que as RSSF sejam maioritariamente instaladas com caracter�sticas
est�ticas ou de pouca mobilidade, a exist�ncia de um modelo de mobilidade poder� possibilitar
a avalia��o dos comportamentos dos protocolos, mediante esta propriedade;
\item[Interface de Visualiza��o] � importante que o ambiente de simula��o possua uma interface de
visualiza��o, permitindo uma percep��o mais f�cil dos comportamentos dos protocolos (ex:
topologia, cobertura), bem como o controlo da simula��o e extrac��o de resultados.
\item[Modelo de Gest�o de Topologia] Um factor que pode influenciar o comportamento de
um protocolo de encaminhamento � a topologia. Como tal, � relevante a exist�ncia deste modelo por
forma a avaliar os protocolos desenhados, perante diferentes topologias.
\end{description}
\subsection{Prowler/JProwler}
 \cite{jprowler}Esta ferramenta resulta da convers�o de um simulador de eventos
discretos\footnote{Fila global, onde s�o inseridos todos os eventos da rede, tratados sequencialmente ou por prioridade.}, Prowler \cite{prowler}, implementado em MATLAB, pela Universidade de
Vanderbilt, para a linguagem Java. Este simulador pode ser
configurado para simular, de forma determin�stica ou probabil�stica. Permite a simula��o, com diversos
n�s, podendo atingir os 5000 (ainda que o n�mero possa ser maior, por raz�es de
performance, este � o valor m�ximo aconselhado), usando diversas
topologias (din�micas/aleat�rias), �s quais se podem sujeitar diversos
algoritmos. 

O JProwler modela os aspectos mais importantes do modelo de comunica��o de uma RSSF. A natureza
n�o-determin�stica da propaga��o r�dio � caracterizada por um modelo de r�dio
probabil�stico simples e preciso, que descreve a opera��o da camada MAC. Possui uma janela de
visualiza��o da topologia da rede. Para o desenvolvimento de aplica��es ou protocolos s�o
disponibilizadas classes base que se podem estender. Est�o presentes dois modelos de r�dio: um de
Gauss, para topologias est�ticas, e outro de Rayleigh, para topologias m�veis.
\subsection{J-Sim}
J-Sim (anteriormente conhecido como JavaSim) � um ambiente de simula��o baseado em componentes
 \cite{jsim}, implementado em Java. N�o foi desenvolvido inicialmente com vista � sua utiliza��o em
RSSF, como � o caso do ambiente SENSE \cite{SENSE}, mas o objectivo de extensibilidade � comum. Este ambiente � amplamente usado e implementa um modelo de rede em camadas.
No entanto, este simulador n�o � o mais adequado para o estudo do desempenho em RSSF, visto que
este � condicionado pelo \textit{hardware}, pelo sistema operativo, pelos protocolos de rede e pelas
aplica��es, assim como pelas optimiza��es espec�ficas entre camadas da pilha de protocolos. Apesar disto, o
J-Sim � um importante ambiente de simula��o, dada a natureza fracamente ligada dos seus componentes,
a qual permite o desenvolvimento e prototipagem de aplica��es. Exige, no entanto, algum
conhecimento profundo da arquitectura, mesmo para a implementa��o de protocolos simples.
\subsection{Freemote}
Fremote � uma ferramenta de emula��o \cite{freemote} distribu�da\footnote{Funciona em rede, com a
possibilidade de ter diversos clientes de visualiza��o, ligados a um servidor central.}, desenvolvida
em Java, utilizada para o desenvolvimento de \textit{software} para RSSF. O emulador suporta
\textit{motes} (Squawk, Sentilla) e plataformas (Java Cards, SunSpot \cite{sunspotworld}), baseados
em Java. Divide a arquitectura em camadas bem definidas por interfaces: Aplica��o, Encaminhamento e
Liga��o de Dados/MAC. Tem um interface gr�fico para configura��o. Suporta experi�ncias de grande
escala (10.000 n�s), incluindo a sua integra��o com n�s reais, baseados em Java. Os principais pontos
negativos s�o: i) O modelo de propaga��o r�dio � muito simples, uma vez que n�o
considera obst�culos entre os n�s; ii) S� existe um modelo de comunica��o
real, limitado a emula��o simples de plataformas espec�ficas (JMote); iii) N�o � orientado para a
an�lise de performance das redes, caracter�stica que pode ser importante no desenvolvimento de
algoritmos para RSSF.
\subsection{ShoX}
A ideia principal deste simulador \cite{shox} � a de proporcionar, de uma forma f�cil e intuitiva, a
implementa��o e desenho de protocolos de rede, modelos de mobilidade, modelos de propaga��o de sinal
ou de tr�fego de rede. Tal como outros simuladores, incorpora um simulador de eventos discretos, que
faz a gest�o de todos os eventos da rede. Todos os conceitos conhecidos no dom�nio das redes sem
fios s�o modelados neste simulador (modelo OSI, pacotes, mobilidade e energia). Uma das vantagens �
a exist�ncia de classes abstractas para reimplementa��o de novos modelos em cada um dos
componentes, facilitando a programa��o de novos protocolos ou de novas funcionalidades. A comunica��o
entre componentes � feita por interm�dio de eventos, ou seja, n�o existe acesso de um componente a
outro. Deve-se destacar o interface gr�fico, que permite operar todas as configura��es
da ferramenta, sem a necessidade de editar directamente os ficheiros de XML.
Para al�m disso, � ainda poss�vel visualizar a topologia de rede e extrair resultados gr�ficos da
simula��o. O facto de o modelo de propaga��o de sinal ser baseado na norma IEEE802.11 dificulta a
adapta��o �s condi��es das RSSF. No entanto, a modularidade do sistema
permite o desenvolvimento de uma camada IEEE802.15.4 para se aproximar da norma
mais recente de comunica��o das RSSF\footnote{N�o cabe no �mbito da disserta��o o desenvolvimento do m�dulo de comunica��o}. A arquitectura deste simulador aproxima-se bastante daquilo que deve ser um
simulador de RSSF, em que as diversas camadas est�o bem definidas.


\newpage
\section{ Discuss�o e Resumo do Trabalho Relacionado}
As RSSF representam um enorme desafio para a investiga��o de sistemas e
protocolos de seguran�a. As caracter�sticas que as tornam uma mais-valia para a opera��o em
ambientes remotos, apresentam-se, simultaneamente, como as suas maiores vulnerabilidades, em termos de
seguran�a. Este paradoxo � contornado com mecanismos de seguran�a inovadores e que se distinguem
dos existentes nas redes convencionais. Assim, passada em revista as diversas
dimens�es abarcadas por esta disserta��o (protocolos de encaminhamento seguro
em RSSF e plataformas de simula��o de RSSF), importa apresentar uma vis�o
cr�tica do trabalho relacionado.

Na \tablename{ \ref{tab:tabela_ataques_contramedidas}} apresenta-se uma vis�o estruturada
das contra-medidas que permitem mitigar ou diminuir o impacte dos ataques nas RSSF. O uso de
criptografia sim�trica � predominante, uma vez que representa uma forma de garantir propriedades, tais
como, confidencialidade, autenticidade e integridade. O uso de fun��es de dispers�o de um sentido
permite verificar a integridade e, aliado ao uso de ``desafios''\footnote{N�meros sequenciais, cuja
sequ�ncia depende da aplica��o de uma fun��o $f^n(x)=C$, $n$ vezes, por forma a obter $C$
(\textit{challenge})}, permite garantir a frescura das mensagens, a um custo computacional reduzido.
� de salientar que as implementa��es destes mecanismos s�o optimizadas para a redu��o dos custos
computacionais e de comunica��o. Esta tabela propicia uma vis�o mais actualizada das contra-medidas
face aos ataques referenciados no modelo de advers�rio, fruto da consolida��o do trabalho
relacionado.
%tabela de ataques e contra-medidas
\begin{table}[H]
 \centering
\begin{tiny}
\begin{tabular}[t]{l|l|p{4cm}}
\hline
Modelos & Ataque & Contramedidas\\ \hline\hline
\multirow{1}{*}{Dolev-Yao}
& Ataque ao meio de comunica��o & Criptografia sim�trica, \textit{One Way Hashing} \\\cline{2-3}
\hline 
%
\multirow{4}{*}{Organiza��o e Descoberta da Rede}
 & Falsifica��o de informa��o de Routing & Autentica��o, \textit{One Way Hashing}\\\cline{2-3}
%%
 & Ataques de \textit{Rushing} & Selec��o aleat�ria de RREQ, autentica��o, verifica��o bidirectional
\\ \cline{2-3}
\hline
%
\multirow{4}{*}{Estabelecimento de Rotas}
 & HELLO flooding & Autentica��o com verifica��o bidirectional(\textit{acknowledge})\\\cline{2-3}
%% 
& Ataques \textit{Sinkhole} & Autentica��o, Distribui��o de chaves \textit{pairwise} \\\cline{2-3}
%% 
& Ataques \textit{Wormhole} & \textit{Packet leaches}, MAC\\\cline{2-3}
%% 
& Ataques \textit{Sybil} & Distribui��o de chaves \textit{pairwise}, selec��o aleat�ria de canais
de r�dio \\
 \hline
%
\multirow{1}{*}{Manuten��o de Rotas} & Ataques de \textit{Backhole} & Defini��o de temporizadores e
mecanismos de confirma��o (ACK) autenticados\\
\hline
%
\multirow{2}{*}{Modelo de Intrus�o}
& Intrus�o& Encaminhamento multi-rota; \textit{One Way Hashing} \\\cline{2-3}
%%
& Replica��o& Certifica��o central; Autentica��o; N�s vizinhos como testemunhas\\\cline{2-3}
\hline

\end{tabular} 
\caption{Tabela de Ataques \textit{vs} Contramedidas}\label{tab:tabela_ataques_contramedidas}
\end{tiny}
\end{table}

Tendo em conta as contra-medidas apresentadas, cabe analisar, comparativamente,
os algoritmos e a capacidade de resistir aos ataques definidos no modelo de
advers�rio. Assim, na \tablename{ \ref{tab:ataques_vs_protocolos}}, est�o
marcados com o s�mbolo \checkmark os ataques defendidos por cada protocolo
estudado, sendo que os marcados com \texttimes\ n�o defendem ou s� o
fazem em condi��es especiais. 

O protocolo SIGF distingue-se dos demais protocolos estudados, particularmente pela sua origem (extens�o
do IGF) e por ser baseado em localiza��o. Esta caracter�stica � espec�fica para determinadas
aplica��es e obriga � exist�ncia de dispositivos especializados nos \textit{motes}. Para al�m disto,
� particularmente indicado para a monitoriza��o de eventos, cuja ocorr�ncia � espa�ada no tempo. Por
ser configur�vel, faz com que se adapte consoante o grau de amea�a, aumentando os custos
de opera��o com o aumento da amea�a.

Os protocolos Clean-Slate e INSENS, n�o necessitam de conhecimento de
localiza��o, diminuindo a complexidade da plataforma de rede. Estes protocolos,
devido �s caracter�sticas que os definem, s�o excelentes candidatos a um estudo
comparativo.  Distinguem-se na utiliza��o da uma esta��o base como
unidade central de encaminhamento, no INSENS. Contrariamente, o Clean-Slate
tem uma abordagem completamente distribu�da. A quest�o que se levanta, no
INSENS, prende-se, essencialmente, com o impacte no consumo energ�tico dos n�s
pr�ximos (a um \textit{hop}) da esta��o base, uma vez que esta encaminhar� todo
o tr�fego da rede. 
%tabela de ataques e protocolos de encaminhamento
{%
\begin{table}[H]
\centering
\begin{tiny}
\begin{tabular}{c|c|c|c|c|c|c|c|c|c|}\cline{2-10}
\mc{1}{c}{\textbf{}} & \mc{7}{|c|}{\textbf{Ataques ao Encaminhamento}}&
\mc{1}{|c|}{\textbf{Intrus�o}} & \mc{1}{|c|}{\textbf{Comunica��o}}\\\cline{1-10}
\mc{1}{|c|}{\textbf{Protocolos}} & \mc{1}{|c}{\textbf{Info. Falsa}} &
\mc{1}{|c|}{\textbf{\textit{Rushing}}}& \mc{1}{c|}{\textbf{HELLO flooding}} &
\mc{1}{c|}{\textbf{\textit{Sinkhole}}} & \mc{1}{c|}{\textbf{\textit{Wormhole}}} &
\mc{1}{c|}{\textbf{\textit{Sybil}}}& \mc{1}{c|}{\textbf{\textit{Blackhole}}} &
\mc{1}{c|}{\textbf{\textit{Intrus�o/Replica��o}}} & \mc{1}{c|}{\textbf{\textit{Dolev-Yao}}} \\\hline
%% linhas da tabela
\mc{1}{|c|}{\textbf{SIGF}} & \checkmark & \checkmark &\checkmark & \texttimes &
\mc{1}{c|}{\texttimes} &\checkmark & \mc{1}{c|}{\checkmark} & \texttimes/\texttimes & \checkmark
\\\hline
%
\mc{1}{|c|}{\textbf{INSENS}} & \checkmark & \checkmark & \checkmark & \checkmark  &
\checkmark  & \checkmark & \checkmark  & \checkmark/\texttimes & \checkmark \\\hline
%
\mc{1}{|c|}{\textbf{Clean-Slate}} & \checkmark & \checkmark & \checkmark & \checkmark &
\checkmark & \checkmark& \checkmark & \checkmark/\checkmark & \checkmark\\\hline
\end{tabular}
\caption{Tabela de Protocolos de Encaminhamento \textit{vs} Ataques
}\label{tab:ataques_vs_protocolos}
\end{tiny}
\end{table}
}%

Ambos os protocolos implementam resili�ncia � intrus�o. Uma das diferen�as � que
o Clean-Slate tem uma ac��o preventiva, correspondente � detec��o dos n�s
maliciosos (replicados). Al�m disso, faz encaminhamento multi-rota, o que
minimiza o impacte dos intrusores que resistam � detec��o. No caso do INSENS,
apenas existe um mecanismo redundante multi-rota. 

Em rela��o aos ataques que podem ser direccionados a cada um dos protocolos, �
importante real�ar o ataque \textit{sybil}, como um caso de dif�cil resolu��o,
quando este deriva de uma intrus�o, em que um atacante obteve as chaves
necess�rias para se poder anunciar com qualquer uma
das identidades que assumir. Assim, a resist�ncia a este ataque, por parte dos
protocolos estudados, n�o leva em conta este modo de intrus�o, porque assume que
cada atacante n�o tem todas as chaves necess�rias para se
autenticar com falsas identidades e que, como tal, pode ser detectado.
\input{tab_simuladores_criterios}
Por fim e sendo a an�lise de ambientes de simula��o um dos focos do trabalho relacionado,
apresenta-se na \tablename{ \ref{tab:Criterios_vs_ambientes}} a sua sistematiza��o, contrapondo os
crit�rios aos ambientes estudados. Como uma primeira nota, deve-se salientar que,
tendo em vista a concep��o de uma plataforma de simula��o, a observa��o dos crit�rios de
\textit{software} para a selec��o de um simulador base teve um peso bastante grande, nomeadamente
no que se refere � capacidade de extensibilidade reflectida na simplicidade.

Perante a necessidade de seleccionar um ambiente que vise alcan�ar os objectivos
da disserta��o, esta selec��o recaiu, no simulador JProwler. A sua simplicidade
� uma mais valia, uma vez que, sendo composto por nove classes, bem
documentadas, � de mais f�cil extensibilidade quanto � implementa��o das
funcionalidades requeridas na plataforma concebida. Cada componente de um n�
sensor � mapeado numa classe abstracta. O modelo de comunica��o � inspirado no
Mica2 com a gest�o de eventos baseada no TinyOS \cite{tinyos}, o que o torna, do
ponto de vista da sua aproxima��o � realidade, bastante vantajoso. � certo que
algumas funcionalidades tiveram de ser implementadas de raiz, como � o caso do
m�dulo energia (existente no ShoX) ou gest�o de topologias (existente no ShoX e
no Freemote), o que permitir� desenvolver modelos que podem ser bem integrados
de forma observar-se melhor as propriedades desejadas. O caso da emula��o
(existente no Freemote) n�o � um requisito mandat�rio da plataforma. Logo, a sua
utiliza��o n�o se apresenta como uma mais-valia, face aos aspectos menos
positivos. O Freemote assenta sobre uma interac��o cliente/servidor, o que torna
dif�cil a depura��o de erros. Possui um modelo de comunica��o demasiado b�sico,
n�o incorporando atenua��es do meio ambiente, o que inviabiliza a sua selec��o.
O J-Sim � realmente uma plataforma poderosa, mas a sua dimens�o obrigaria a um
esfor�o adicional, demasiado grande, na medida em que necessitaria de ver
desenvolvido muitos dos seus m�dulos de raiz.

Desta forma, perante a maior complexidade de algumas plataformas, o JProwler
constitui uma base simples, eficiente e, simultaneamente, com capacidade para
ser enriquecida, de forma a integrar a plataforma de simula��o. Ainda assim,
alguns dos conceitos presentes nos outros simuladores prestaram significativos
contributos conceptuais para o resultado final da plataforma concebida no
�mbito desta disserta��o.

%No �mbito dos modelos de energia desenvolvidos para sistemas de simula��o
%\cite{powertossim,omnet++,swanenergymodel}, os estudos conhecidos apresentam
%diversas abordagens mais ou menos relaxadas no que respeita ao c�lculo do
%consumo energ�tico. Por um lado, alguns modelos, por estarem muito pr�ximos da
%plataforma fisica, como � o caso do PowerTossim\cite{powertossim}, conseguem
%obter um valor aproximado dos custo dos ciclos de rel�gio permitindo obter
%alguma acuidade no valores referentes a este tipo de eventos. Outros sistemas
%como o caso do modelo OMNET++\cite{OMNET++} come�am por assumir 

\chapter{Abordagem � fase de elabora��o da disserta��o}\label{cap:elabora��o}
Tendo sido abordadas as tem�ticas relacionadas com a problem�tica da seguran�a numa
RSSF, importa ent�o, definir uma estrat�gia para a concep��o de uma plataforma que vise a an�lise e
avalia��o de protocolos de encaminhamento, principalmente os concebidos com requisitos de
seguran�a. Neste cap�tulo apresenta-se, de forma preliminar, as fases da elabora��o da
disserta��o referentes � concep��o dos modelos que suportam a arquitectura da plataforma.
Adicionalmente, apresenta-se uma prova de conceito, referente � utiliza��o da plataforma, com a
implementa��o de dois protocolos Clean-Slate e INSENS e o conseguinte estudo comparativo.
\section{Desenho e concep��o da plataforma de simula��o}
\subsection{Consolida��o da avalia��o de ambientes de simula��o e a sua
incorpora��o}\label{sec:elab_aval_amb_sim}
Do estudo das plataformas de simula��o existentes e apresentadas neste relat�rio, surgir� um motor
de simula��o com os modelos base de um ambiente de RSSF (comunica��o, gerador de
eventos discretos e plataformas de sensores). Assim, ir� proceder-se �
escolha de um simulador base com estas caracter�sticas, j� fundamentadas e discutidas no cap�tulo
anterior. No entanto, a fase de elabora��o da disserta��o dever� contar, inicialmente, com o 
aprofundamento da avalia��o dos ambientes de simula��o apresentados, com o intuito de certificar a
selec��o do simulador base. Uma mais valia deste estudo ser� o
aproveitamento de caracter�sticas presentes em diversos ambientes e que possam ser aproveitadas para
a implementa��o na plataforma final. Note-se que, a avalia��o ter� que
ter em aten��o sempre o objectivo do trabalho de modo a que, o tempo despendido a apreender
determinada arquitectura n�o seja superior � poss�vel implementa��o de raiz.
\subsection{Apresenta��o preliminar da arquitectura da plataforma de
simula��o}\label{sec:apresent_arch}
A vis�o mais simples para representar a plataforma � sobre a forma
de uma pilha de servi�os. Como se pode ver na \figurename\ \ref{fig:sim_arch}, os principais
servi�os s�o: i) Mecanismo de Gera��o de Topologias; ii) Mecanismo de Gest�o de
Consumo de Energia; iii) Mecanismo de Injec��o de Faltas/Ataques ao Encaminhamento; iv) Mecanismo de
Configura��o; v) Mecanismo de Visualiza��o e Controlo de Simula��o.
\begin{figure}[H]
\centering
\includegraphics[height=7cm,width=6cm]{Simulador_arquitectura.png}
\caption{Arquitectura de Simula��o} \label{fig:sim_arch}
\end{figure}
\subsubsection{Implementa��o do Mecanismo de Configura��o}\label{sec:impl_conf}
Para dotar a plataforma de maior flexibilidade, a exist�ncia de um componente gestor de
configura��es revela-se importante. Este componente, � transversal a toda 
plataforma. Pois, cada parte ter� especificidades pr�prias que ser�o geridas por
este componente. Para que as parametriza��es possam ser persistentes e port�veis adoptar-se-� a
tecnologia XML para defini��o dos ficheiros de configura��o da plataforma. As principais
funcionalidades que se prev�em existir v�o desde as configura��es dos par�metros do simulador base,
at� � configura��o de cada uma das simula��es, que se pretende estudar, como forma de possibilitar a
repeti��o de experi�ncias nas mesmas condi��es.
\subsubsection{Implementa��o do Mecanismo de Gera��o de Topologias}\label{sec:impl_topo}
As RSSF, normalmente, s�o caracterizadas por diferentes formas de distribui��o
dos n�s sensores. Estas distribui��es podem ser essencialmente divididas em dois modelos:
aleat�rio e estruturado. Assim, para que se consiga acrescentar mais um grau de liberdade �s
caracter�sticas que se pretende observar na an�lise de um protocolo de encaminhamento, ser�
fornecido um componente cuja fun��o � gerar topologias na rede. Pois, sabe-se que a
topologia da rede pode influenciar o comportamento de um protocolo. Pretende-se que este
componente, possibilite a extens�o para novas topologias (espec�ficas para determinada
simula��o).
\subsubsection{Implementa��o do Mecanismo de Gest�o de Consumo de Energia}\label{sec:impl_energia}
As caracter�sticas fundamentais que se dever�o implementar neste componente s�o: a 
defini��o de uma API para o desenvolvimento de novos modelos consoante as necessidades
da simula��o; a possibilidade de introduzir par�metros relacionados com o consumo por parte dos
sensores. Associadas a este componente estar�o funcionalidades que permitir�o a
recolha de informa��o em tempo real dos consumos da rede, quer no seu total, quer individualmente,
em cada sensor. Desta forma, este � um dos componentes de maior import�ncia, uma vez que um dos
indicadores que se pretende observar na an�lise de protocolos de encaminhamento, � o impacto sobre o
tempo �til de opera��o da rede, quer em condi��es de funcionamento normais, quer em condi��es
de ataque efectivo, tempo este que est� dependente da energia.
\subsubsection{Implementa��o do Mecanismo de Injec��o de Falhas / Ataques ao Encaminhamento
}\label{sec:impl_ataques}
Sendo o tema central da futura disserta��o o estudo de protocolos de encaminhamento seguros em
RSSF, este componente � o de maior import�ncia nesta plataforma e por diferentes ordens de raz�es:
i) N�o existe nenhum sistema de simula��o que permita a indu��o de ataques de forma simples e
intuitiva, consubstanciado-se num contributo para a inova��o; ii) Dever� ser suficientemente
flex�vel para se adaptar � l�gica de cada algoritmo; iii) Poder� permitir a muta��o de c�digo em
tempo de execu��o da simula��o, por forma a alterar comportamentos do protocolo; iv) Idealmente
dever� permitir a acrescentar mais modelos de ataques, dos j� tipificados neste relat�rio ou outros
que venham a ser desenvolvidos.
\subsubsection{Implementa��o do Mecanismo de Visualiza��o e Controlo de Simula��o
}\label{sec:impl_visual}
Como n�o poderia deixar de ser existe a necessidade de dotar a plataforma de um ambiente de
opera��o. Como tal, � necess�rio implementar um componente correspondente � visualiza��o gr�fica de
toda a simula��o, bem como a possibilidade de controlar par�metros de execu��o. Pretende-se
desenvolver um ambiente gr�fico integrado que permita a configura��o da plataforma, a configura��o
e visualiza��o das simula��es e a extrac��o de resultados relacionados com as medidas que se
pretendem avaliar: energia, fiabilidade, cobertura, principalmente sobre a forma de gr�ficos.
\subsubsection{Avalia��o da Solu��o}\label{sec:aval_solucao}
Uma vez que a contribui��o efectiva para a componente de investiga��o de
protocolos de encaminhamento seguros em RSSF, ser� obtida com a concep��o de uma plataforma de
simula��o que suporte o estudo e a an�lise desta problem�tica, importa sujeit�-la a uma avalia��o
prim�ria que permita comprovar a sua utilidade e/ou identificar eventuais lacunas neste dom�nio.
Assim, tendo esta avalia��o em vista, pretende-se contribuir com o estudo dos protocolos de
encaminhamento seguro referidos. Para isso, definem-se duas fases complementares na elabora��o da
tese, uma que compreende a implementa��o de dois protocolos, seguidos da fase de experimenta��o
usando as funcionalidades da plataforma. No final, ser� poss�vel salientar as caracter�sticas de
cada protocolo implementado analisadas � luz desta plataforma. 
\subsection{Implementa��o de Protocolos de Encaminhamento Seguro em RSSF}
\subsubsection{Fase de desenho dos algoritmos baseado nas especifica��es}\label{sec:impl_protocolos}
No in�cio desta fase, ser� necess�rio re-aprofundar o funcionamento de cada algoritmo a implementar,
conhecer e identificar cada mecanismo, especificado, de modo a que se possa, dentro do poss�vel,
generalizar opera��es ou interfaces com vista a reutiliza��o em outros algoritmos. Assim
sendo, esta fase exigir� uma aprendizagem/conhecimento de cada algoritmo contribuindo
tamb�m para a especializa��o neste dom�nio.
\subsubsection{Fase de avalia��o dos algoritmos}\label{sec:aval_protocolos}
Recorrendo �s ferramentas disponibilizadas pela plataforma dever� ser poss�vel, no final da
implementa��o, sistematizar as simula��es por forma a extrair resultados, que por si s�,
caracterizem os algoritmos em mat�ria de seguran�a e a sua correc��o em determinados par�metros que
se julgam ser fundamentais no estudo de protocolos de encaminhamento. Alguns dos quais, se indicam a
seguir: i) correc��o do protocolo; ii) an�lise do consumo de energia; iii) fiabilidade/entrega de
mensagens; iv) correc��o dos eventos; v) lat�ncia. Estes contribuem tamb�m, para a avalia��o
da usabiliadade da plataforma na avalia��o/compara��o de protocolos de encaminhamento seguro em
RSSF.
\chapter[Plano de trabalho]{Plano de Elabora��o da Tese}
\label{cap:plano}
A elabora��o da tese realizar-se-� durante o 2� semestre de 2009/2010, iniciando a 22 de Fevereiro
de 2010. O plano apresentado estabelece cinco grandes actividades: an�lise, desenvolvimento,
prova de conceito, avalia��o e relat�rio, como se apresenta na \figurename{\ref{fig:gantt}}.
\begin{figure}[H]
\centering
\includegraphics[height=14cm,width=16cm]{gantt.png}
\caption{Plano da Disserta��o} \label{fig:gantt}
\end{figure}

Apresenta-se em seguida uma breve descri��o de cada uma das actividades.
\begin{description}
\item[An�lise]
Esta actividade corresponde � revis�o de bibliografia complementar, como forma de aprofundar o
estudo de problem�tica, avalia��o do simulador base como preconizado na sec��o
\ref{sec:elab_aval_amb_sim}. In�cia-se tamb�m o desenho da plataforma, consistindo numa vis�o mais
t�cnica da solu��o, com a defini��o formal de algoritmos, interfaces e o modelo de interac��o
dos componentes da plataforma. Esta �ltima tarefa permitir� in�ciar a fase de desenvolvimento de
forma mais s�lida.
\item[Desenvolvimento]
Esta actividade corresponde � concep��o e implementa��o da arquitectura da plataforma como
apresentada na sec��o \ref{sec:apresent_arch}, em que cada um dos componentes corresponder� �s
seguintes tarefas: Integra��o do Simulador Base, M�dulo de Configura��o (sec��o
\ref{sec:impl_conf}), M�dulo de Gest�o de Topologias (\ref{sec:impl_topo}), M�dulo de Gest�o de
Energia (sec��o \ref{sec:impl_energia}), M�dulo de Visualiza��o (sec��o \ref{sec:impl_visual}) e
M�dulo de Injec��o de Ataques (sec��o \ref{sec:impl_ataques}).
\item[Prova de Conceito]
Esta actividade corresponde � implementa��o dos algoritmos de encaminhamento seguro propostos (ver
sec��o \ref{sec:impl_protocolos}):  INSENS e Clean-Slate, que dever� ser precedida de um estudo mais
aprofundado das particularidades de cada um. Consequentemente, cada protocolo ser� sujeito a um
modelo de ataques que permitir�o avaliar os comportamentos como forma de os mapear nas
contribui��es esperadas para esta disserta��o.
\item[Avalia��o]
Esta actividade preconiza a avalia��o dos protocolos implementados usando as ferramentas da
plataforma como referido na sec��o \ref{sec:aval_protocolos}. No entanto, esta avalia��o permitir�,
tamb�m, retirar conclus�es acerca da usabilidade da plataforma (sec��o \ref{sec:aval_solucao}) e o
grau de sucesso dos objectivos pretendidos que passam pela capacidade de estudo de
protocolos de encaminhamento em RSSF em geral, e em particular, os com preocupa��es de seguran�a.
\item[Relat�rio]
Esta actividade corrresponde � escrita da disserta��o que poder� decorrer em paralelo com a
avalia��o e eventualmente com a prova de conceito. Que culminar� com a entrega da disserta��o at� �
data limite.   
\end{description}

\chapter[Prova de Conceito da Plataforma]{Prova de Conceito da Plataforma}

Uma vez implementada a plataforma de simula��o � necess�rio submete-la a uma
utiliza��o dirigida para o fim para o qual foi concebida. Tendo isto em
aten��o, este cap�tulo apresenta a implementa��o de dois protocolos de
encaminhamento de dados em RSSF. O primeiro trata-se de um protocolo simplista,
encaminhamento por \textit{inunda��o }, que visa, fundamentalmente, demonstrar a
facilidade de implementa��o de um primeiro protocolo e ao mesmo tempo servir de
termo de compara��o com um protocolo seguro. O segundo protocolo, � um
protocolo de encaminhamento seguro de refer�ncia, o INSENS\cite{INSENS}.
Esta implementa��o, visa, n�o s� provar que � poss�vel implementar, avaliar e
experimentar protocolos de especifica��o mais complexa, mas tamb�m contribuir
com uma an�lise critica deste protocolo em crit�rios de seguran�a e no seu
impacte nas propriedades que se podem avaliar com recurso � plataforma de
simula��o.

Com a implementa��o destes dois protocolos, este cap�tulo permite,
adicionalmente, descrever uma metodologia para a implementa��o de
protocolos de encaminhamento usando a API concebida. Deste modo, � poss�vel
retirar o m�ximo partido da plataforma com o objectivo de estudar estes
protocolos.

\section{Implementa��o de Protocolo de encaminhamento gen�rico}
Como j� foi referido para efeito de compara��o de algumas das propriedades em
avalia��o foi desenvolvido um primeiro protocolo mais simples que implementa o
encaminhamento por inunda��o (\textit{flooding}). Este protocolo tem grande
complexidade de funcionamento, uma vez que a �nica restri��o para a propaga��o
de mensagens � a garantia de n�o repeti��o de mensagens. 

A implementa��o de um qualquer protocolo pode ser divida em duas componentes:
1) a componente l�gica, associada ao funcionamento do protocolo 
2) a componente arquitectural, associada � integra��o na plataforma.
\subsection{Componente L�gica}
Para esta componente � necess�rio identificar as entidades principais de
especifica��o do protocolo. Entidades estas que ser�o comuns a qualquer
protocolo que se pretenda implementar, e que se apresentam a seguir:

\begin{enumerate}
 \item Message: Entidade que representa a unidade m�nima de envio/recep��o.
Cont�m uma estrutura gen�rica suficiente para gest�o da informa��o manipulada
de forma transparente pelo simulador. No entanto, trata a informa��o
transportada como um vector de bytes assemelhando-se com a realidade de um
pacote de comunica��o;
\item Node: Cada protocolo dever� ter a especializa��o de uma entidade do tipo
Node. Permitir� que o comportamento desta entidade possa ser especifica para o
protocolo implementado (por exemplo, altera��o de cores, tamanho ou
inicializa��o de dados);
\item RoutingLayer: A entidade principal uma vez que ser� esta que cont�m a
l�gica do protocolo e que dever� respeitar algumas especifica��es da API para
que a sua integra��o com o simulador seja adequada;
\item Application: Representa a camada aplicacional que ser� suportada em
termos de encaminhamento pelo protocolo implementado.
\end{enumerate}
Na figura seguinte observa-se a estrutura do protocolo de inunda��o criada com
base na l�gica inerente ao protocolo. 
\begin{figure}[H]
\centering
\includegraphics[height=3cm,width=5cm]{flooding_dir.png}
\caption{Estrutura l�gica de um protocolo}
\label{fig:protocol_struct}
\end{figure} 


\subsection{Componente Arquitectural}
Na componente de integra��o � necess�rio criar uma entidade que integra/define
quais as camadas existentes num sensor gen�rico de um protocolo essa entidade �
mapeada a partir de um padr�o de \textit{factory}. Padr�o este que � consumido
pela plataforma para cria��o de sensores a quando da sua distribui��o na �rea
de trabalho. Esta entidade durante a cria��o dos n�s ser� ainda complementada
com informa��o referente a par�metros do modelo de energia.
Na figura seguinte pode-se observar a simplicidade desta entidade e a forma de
integra��o.
\begin{figure}[H]
\centering
\includegraphics[height=4.5cm,width=10cm]{node_factory_setup.png}
\caption{Defini��o de uma NodeFactory espec�fica}
\label{fig:node_factory_setup}
\end{figure} 
\section{Implementa��o de Protocolo de Encaminhamento Seguro em RSSF}
\subsection{Fase de desenho dos algoritmos baseado
nas especifica��es}\label{sec:impl_protocolos}
No in�cio desta fase, foi necess�rio re-aprofundar o funcionamento do
algoritmo a implementar, conhecendo e identificando cada mecanismo/t�cnica
especificados, de modo a que se pudesse, dentro do poss�vel, generalizar
opera��es ou interfaces, com vista � reutiliza��o para outros algoritmos,
contribuindo para a API gen�rica de implementa��o de protocolos de
encaminhamento.



  
\chapter{Avalia��o}
Neste cap�tulo apresentam-se os resultados das experi�ncias realizadas para
prova de conceito da plataforma e por conseguinte para avalia��o do protocolo
implementado. Apresenta-se em primeiro lugar a metodologia seguida para a
execu��o dos testes que permitem efectuar a avalia��o, seguida da apresenta��o
dos resultados em forma gr�fica seguidas de uma an�lise cr�tica.
 
\section[Metodologia]{Metodologia} \label{sec_metod_testes}
Para a avalia��o do protocolo INSENS foram utilizadas as ferramentas
disponibilizadas pela plataforma de simula��o. Estas ferramentas assentam nos
pressupostos j� apresentados anteriormente e seguiram um modelo que permite a
compara��o dos resultados obtidos com outros protocolos que venham a ser alvo
de estudos comparativos. Assim, foi necess�rio desde logo efectuar um conjunto
de testes de aferi��o para permitir chegar a configura��es de testes que
permitam observar/analisar as propriedades que se pretendem ver estudadas.
Estas propriedades s�o cobertura, lat�ncia, fiabilidade e consumo de energia. O
estudo � realizado sobre o protocolo INSENS com e sem ataques ao encaminhamento.
Desta forma poder-se-� avaliar o impacte que determinado ataque tem, ao n�vel
destas propriedades, no protocolo INSENS. O protocolo de inunda��o � um exemplo
de protocolo com alta resili�ncia � falha e com caracter�sticas de alta
cobertura e fiabilidade. Isto motivado pelo elevado n�mero de replicas de
mensagens circulam na rede. O que parece ser uma vantagem, apresenta-se como uma
vulnerabilidade uma vez que os consumos de energia representam valores muito
altos que deduzem em muito o tempo de opera��o de uma rede. No que respeita a
seguran�a caso seja implementados os mecanismos de protec��o garante-se um
aumento na confidencialidade, integridade e autenticidade das mensagens,
representando no entanto mais um custo energ�tico adicional. 

Tendo como ponto de partida algumas caracter�sticas documentadas do protocolo
INSENS numa fase inicial realizaram-se algumas experi�ncias para se verificar o
comportamento do protocolo e com isso permitir aferir melhores condi��es para a
realiza��o dos testes. Por exemplo, o protocolo INSENS evidenciou a necessidade
da exist�ncia de uma grande densidade de sensores por forma a permitir obter um
grau de estabilidade do protocolo adequado aos testes. Com isto foi necess�rio
encontrar dimens�es adequadas para os campos de distribui��o de sensores por
forma a garantir estes valores determinados.

Estes valores de densidade carecem tamb�m de uma an�lise com vista a explicar
alguns dos fen�menos que ocorrem ao longo das experi�ncias.
Para a parametriza��o de cada experi�ncia foram tidas em considera��o as
seguintes propriedades:

\begin{enumerate}
 \item Dimens�o do campo: determinou-se para cada experi�ncia a dimens�o
adequada para garantir um n�mero espec�fco de vizinhos que permitisse comparar
executar os testes em condi��es semelhantes.
 \item Quantidades de sensores distribu�dos: foram executados testes com
500,750,1000,1500,2000,2500,3000 sensores. Estes valores j� permitem ter uma
boa indica��o em condi��es de escalabilidade, sem que as experi�ncias demorem
um tempo exagerado.
 \item N�mero de sensores atacantes: Foi seleccionado um ataque ao
encaminhamento que se sabe que, � partida, ter� um efeito efectivo nas
propriedades. O n�mero de sensores comprometidos � calculado em percentagem de
5\%,10\%,15\% e 20\% face ao n�mero de sensores est�veis.
 \item N�mero de sensores emissores: Representando os sensores geradores de
eventos, ap�s diversas execu��es sistem�ticas que foram dos 5\% at� ao 80\%
optou-se por um valor de 60\% de sensores est�veis como sensores emissores de
mensagens.
 \item Esta��o base receptora: a esta��o base apresenta-se como a receptora dos
eventos ocorridos na rede.
 
\end{enumerate}


\subsection{Cobertura da Rede}
\paragraph{} Pretende-se testar para o protocolo qual a
percentagem de n�s cobertos na rede. Uma vez que o
simulador introduz a no��o de n� est�vel este valor � directamente determinado
ap�s a fase de \textit{setup} do protocolo de encaminhamento. Ser� poss�vel
extrair o grau de conectividade do protocolo, ou seja, num determinado campo de
distribui��o de sensores qual a percentagem de sensores que ap�s a sua execu��o
conseguem participar no encaminhamento de mensagens. Para al�m desta no��o de
cobertura considera-se tamb�m como m�trica a observar a percentagem de n�s que
no envio de mensagens para a esta��o base conseguem efectivamente fazer chegar
pelo menos uma das mensagens.
\paragraph{} A primeira no��o de conectividade � obtida pelo pr�prio
RoutingLayer que se encarrega de contabilizar os n�s que est�o est�veis. A
segunda no��o � obtida pelo envio intervalado de mensagens para a esta��o base
por cada um de n�s emissores.
\paragraph{Porque � assim que se avalia}
\paragraph{Quais os resultados}
INSERIR O GRAFICO

\paragraph{Conclus�es dos testes}
Coment�rio ao grafico

\subsection{Fiabilidade da Rede}
\paragraph{} Pretende-se avaliar qual o grau de fiabilidade de um protocolo no
encaminhamento de dados, ou seja, uma vez enviadas as mensagens pelos sensores
emissores ou geradores de eventos, quantas destas mensagens chegam � esta��o
base, extraindo-se desta ac��o um indicador da taxa de fiabilidade de
um protocolo esteja este ou n�o sujeito a um ataque.
\paragraph{} Cada n� emissor enviar� um n�mero de mensagens intervaladas para a
esta��o base, findo o teste s�o contabilizadas as mensagens que foram recebidas
face �s enviadas, obtendo-se assim a taxa de fiabilidade de opera��o do
protocolo.
\paragraph{Porque � assim que se avalia}
\paragraph{Quais os resultados}
INSERIR O GRAFICO

\paragraph{Conclus�es dos testes}
Coment�rio ao grafico

\subsection{Lat�ncia da Rede}
\paragraph{} Pretende-se avaliar o indicador de lat�ncia das mensagens enviadas
pelos sensores emissores para a esta��o base. Este indicador baseia-se na
contagem do n� de \textit{hops} que uma mensagem efectua entre origem e
destino. 
\paragraph{} Cada n� emissor enviar� um n�mero de mensagens intervaladas para a
esta��o base, findo o teste s�o contabilizados os valores m�nimo, m�ximo e
m�dio de lat�ncia de uma mensagem na rede.
\paragraph{Porque � assim que se avalia}
\paragraph{Quais os resultados}
INSERIR O GRAFICO

\paragraph{Conclus�es dos testes}
Coment�rio ao grafico


\subsection{Consumo de energia}
\paragraph{} No que se refere a energia a an�lise efectuada de diversos
�ngulos. Primeiro observam-se os resultados decorrentes da execu��o de um
teste, como os anteriores pontos. Segundo Observa-se o perfil de consumo de um
protocolo na rede pela observa��o de ``um mapa de calor'' para determinar as
�reas de maior consumo. Por fim observa-se, num protocolo, os consumos
realizados nos diversos eventos de consumo de energia e observa-se a
distribui��o do consumo de energia pelas diferentes fases de um protocolo. 
\paragraph{} A energia consumida em cada teste executado ser� apresentada com
um valor global do teste e o valor m�dio por sensor. O ``mapa de calor'' de
energia mostra que face ao consumo existente quais as �reas identificadas como
cr�ticas.
\paragraph{Porque � assim que se avalia}
\paragraph{Quais os resultados}
INSERIR O GRAFICO

\paragraph{Conclus�es dos testes}
Coment�rio ao grafico
\chapter{Conclus�es e trabalho futuro}

% ...
% \include{capituloN}

%% Parte ps-texto principal
\backmatter

% Apndices. Comente se no houver apndices
\appendix

%  aconselhvel ter um apndice por ficheiro
% "apendice1.tex", "apendice.tex", ... "apendiceM.tex" e depois
% inclu-los com:
% \include{apendice1}
% \include{apendice2}
% ...
% \include{apendiceM}

% Bibliografia. Deve utilizar o BibTeX a partir de um arquivo, ex: "mybib.bib".
\bibliographystyle{plain}
\bibliography{mybib}

%% Fim do documento
\end{document}