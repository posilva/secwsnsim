\section{Modelo de Advers�rio, Ataques ao Encaminhamento e
Contra-medidas} \label{sect:sec_mod_adversario_ataq_contramedidas}
\subsection{Arquitectura de Servi�os de Seguran�a em
RSSF} \label{sect:subsec_arq_security_wsn}
Muitas vezes considera-se que num sistema seguro, a seguran�a � um
�nico componente da arquitectura do sistema\cite{SECURITY_IN_WSN_PERRIG}.
Todavia, esta vis�o � redutora, pois, para se conseguir um sistema seguro, �
necess�rio que a seguran�a esteja integrada em cada um dos componentes, se
assim n�o se verificar cada componente pode corresponder a uma amea�a a partir
da qual se pode desencadear um ataque. \\
Os desafios propostos pelas RSSF s�o espec�ficos da pr�pria tecnologia, assim
sendo os mecanismos tradicionais de seguran�a podem n�o ser suficientes. Com
isto surgem tecnologias que visam corresponder aos requisitos de seguran�a das
RSSF. Assim, nesta sec��o pretende-se apresentar uma vis�o sobre os requisitos
de seguran�a de uma RSSF e como estes s�o atendidos usando alguns servi�os de
seguran�a mais populares.

\subsubsection{Requisitos de seguran�a de uma RSSF}
Embora se apresente os requisitos de seguran�a de uma RSSF, estes requisitos
podem variar consoante as especificidades da aplica��o que a rede visa
suportar. Por exemplos se se tratar de uma aplica��o de monitoriza��o de sa�de
uma pessoa, o �bvio � proteger a privacidade da pessoa, mas se se tratar de uma
aplica��o de monitoriza��o de dados ambientais este requisito pode ser
relaxado.\\
De seguida apresentam-se, de forma gen�rica, os principais requisitos de
seguran�a de uma RSSF:
\begin{description}\addtolength{\itemsep}{-.50\baselineskip}
 \item[Autentica��o]
 Considerando que as RSSF usam um meio de comunica��o partilhado, � necess�rio
recorrer � autentica��o para garantir a detec��o de mensagens alteradas ou
injectadas no sistema com o intuito de obter acesso a alguma opera��o ou
informa��o que � restrita a utilizadores n�o
autorizados\cite{SECURITY_IN_WSN_PERRIG}. No que respeita a autentica��o o
requisito pode ter dois sentidos: autentica��o da origem das mensagens
(garantindo que a mensagem � gerada por quem se apresenta como sendo a origem)
e autentica��o dos dados (garantindo que os dados recebidos s�o exactamente os
enviados). Note-se que a implementa��o de criptografia assim�trica pode
contribuir para a garantia desta propriedade, mas ainda existe muito esfor�o a
desenvolver neste campo dadas as limita��es das RSSF e as exig�ncias
computacionais e energ�ticas destes mecanismos.
 \item[Confidencialidade]
 Sendo a RSSF uma infraestrutura baseada fundamentalmente na dissemina��o de
dados recolhidos apartir de sensores que se encontram distribuidos em ambiente
n�o controlado e , normalmente, de f�cil acesso,  pode ser necess�rio garantir a
confidencialidade dos dados que circulam na rede. Assim, o uso de mecanismos de
criptografia � o mais usado para protec��o de dados em comunica��o
ponto-a-ponto. O uso de criptografia por si s� pode n�o ser suficiente, uma vez
que um ataque pode fazer an�lise dos dados que circulam na rede (fazendo uma
an�lise de padr�es) e com este conhecimento violar as chaves que servem de base
ao mecanismo. Desta forma, a utiliza��o de algoritmos de encripta��o fiaveis
(ex:AES, ECC) para garantir um determinado n�vel de seguran�a, para isso existe
a necessidade de partilhar chaves de sess�o por todos os \textit{end-points} e
como tal deve-se recorrer a esquemas de distribui��o de chaves.
 \item[Disponibilidade]
Para que consiga atingir um bom n�vel de disponibilidade � necess�rio garantir
que uma RSSF funcione normalmente durante todo o seu tempo de implementa��o. Os
ataques de nega��o de servi�o (Denial of Service - DoS) s�o os mais frequentes
para atingir a disponibilidade de uma rede. Para al�m de mecanismos que evitem
a nega��o de servi�o, � preciso garantir que a forma de degrada��o da rede ( na
presen�a de um ataque ) seja controlada e que a degrada��o v� sendo t�o grande
quanto maior sejam o n�mero de n�s afectados.
 \item[Integridade]
Durante a comunica��o a integridade garante que os dados recebidos pelo n�
receptor n�o foram alterados por um advers�rio durante a transmiss�o. Em alguns
casos esta propriedade � garantida juntamente com a autentica��o, usando
mecanismos que permitam verificar ambos apenas com um mecanismo, poupando assim
recursos, que s�o escassos nas RSSF. Por exemplo, o uso de HMAC's � vulgar uma
vez que a chave � usada para autenticar a origem da mensagens e para garantir a
integridade da mensagem, ou seja se a mensagem for alterada a origem n�o poder�
ser correctamente verificada, pelo que a mensagem acaba por ser
ignorada\cite{SPINS}.
\item[Frescura]
A frescura dos dados de uma mensagem implica que estes sejam recentes
garantindo que esta mensagem n�o � antiga e n�o foi reenviada por um qualquer
advers�rio. \cite{SPINS} Podem-se considerar dois tipos de frescura: frescura
fraca (garantindo ordem parcial e sem informa��o do desvio de tempo, usada para
 as medi��es dos sensores) e frescura forte (garante ordem total em cada
comunica��o permitindo a estima��o do atraso, usada para a sincroniza��o de
tempo).
 \end{description}
\subsubsection{Servi�os de Seguran�a}
\begin{description}\addtolength{\itemsep}{-0.5\baselineskip}
\item[TinySec\cite{Karlof2004}]
TinySec � uma arquitectura de seguran�a para protec��o ao n�vel de liga��o de
dados em RSSF. O objectivo principal, para o qual foi desenhado, � providenciar
um n�vel adequado de seguran�a com o minimo consumo de recursos. Os servi�os de
seguran�a  disponibilizados s�o: autentica��o de dados (com a utiliza��o de
\textit{Message Authentication Codes}(MAC), no caso CBC-MAC\footnote{Cipher
Block Chaining - Message Authentication Code (CBC-MAC))}) e confidencialidade 
(a encripta��o � implementada com o recursotamb�m ao CBC-MAC).
 Para se adaptar �s RSSF o MAC tem um comprimeito de 4 bytes ao contr�rio dos 8 ou 16 habituais.
 Uma vez que neste mecanismo as propriedades de seguran�a est�o determinadas pelo comprimento do MAC
os autores afirmam que para uma RSSF � suficiente uma vez que um atacante teria de efectuar $2^{31}$
tentativas, que demorariam cerca de 20 meses a realizar com um canal de 19.5Kbs.
Note-se que a frescura das mensagens n�o foi endere�ado nesta arquitectura uma
vez que os autores consideram como demasiado exigente em termos de recursos o
que contrariaria o objectivo inicial da arquitectura.
%%%%%%%%%%%%%%%%%%%%%%%%%%%%%%%%%%%%%%%%%%%%%%%
 \item[MiniSec\cite{Luk2007d}]
Minisec � uma camada de rede concebida para possuir o melhor dos dois mundos: baixo consumo de
energia e alta seguran�a. Esta arquitectura tem dois modos de opera��o: uma baseado para
comunica��o \textit{unicast} (MINISEC-U) e outro para \textit{broadcast} (MINISEC-B). Sendo que a
segunda n�o necessita de manter o estado por cada emissor por forma a proteger o reenvio escalando
para grandes redes.
%%%%%%%%%%%%%%%%%%%%%%%%%%%%%%%%%%%%%%%%%%%%%%%
\item[SPINS\cite{SPINS}]
� um conjunto de protocolos de seguran�a, constitu�do por dois componentes
principais SNEP\footnote{Secure Network Encryption Protocol} \cite{SNEP} e
${\mu}$TESLA \cite{MICROTESLA}. O primeiro, fornece servi�os de autentica��o e
confidencialidade entre dois pontos de comunica��o, encriptando as mensagens e
protegendo-as com um MAC. O SNEP gera diferentes chaves, de encripta��o, que 
derivam de uma chave mestra partilhada entre os dois n�s, ainda � incluido um
contador nas mensagens para garantir a frescura. A encripta��o � realizada com
o modo CTR\footnote{\textit{Counter Mode}} e a autentica��o com CBC-MAC. O
segundo componente,o ${\mu}$TESLA,  � um servi�o de autentica��o de
\textit{broadcast}, que evita a utiliza��o de mecanismos, mais exigentes, de
criptografia assim�trica, recorrendo a critografia sim�trica, autenticando as
mensagens com um MAC,
%%%%%%%%%%%%%%%%%%%%%%%%%%%%%%%%%%%%%%%%%%%%%%%
\item[Sistemas de distr�bui��o de chaves\cite{eschenauer_key-management_2002}]
Confidencialidade e autentica��o s�o aspectos cr�ticos das redes de sensores
com o objectivo de prevenir que um advers�rio compromenta a seguran�a de um
sistema. Devido � natureza \textit{ad-hoc} das redes, a comunica��o
intermitente e os recursos limitados, os mecanismos de
gest�o de chaves e autentica��o de grupo tornam-se dificil de alcan�ar.
Alguns esquemas para distribui��o e gest�o de chaves tem vindo a ser
desenvovidos pela investiga��o com vista a compreender as especificidades das
RSSF.

%%%%%%%%%%%%%%%%%%%%%%%%%%%%%%%%%%%%%%%%%%%%%%%
\item[ZigBee\cite{falta}]

\end{description}

\subsection{Modelo de Advers�rio} \label{sect:sec_mod_adversario_serv_seg}
Quando se tratam de quest�es de seguran�a, qualquer se seja o seu dom�nio,
existe uma primeira pergunta que cumpre fazer: ``quais s�o as amea�as/ataques a
que est� sujeito o objecto que se pretende manter seguro?''. Esta pergunta
possibilita, desde logo, encetar uma caminhada que visa a identifica��o de quais
os poss�veis atacantes, que capacidades estes possuem, quais os meios e modos
que estes podem utilizar e em que momento o ataque se pode desencadear. Esta
abordagem, um tanto ou quanto generalista, � suficiente para ilustrar a forma
como se pretende orientar o estudo e com isto apresentar, nas mesmas vertentes,
o modelo de advers�rio que enforma este trabalho.
 \subsubsection{Modelo de Dolev-Yao}\label{sect:subsec_dolev_yao}
Um dos modelos de advers�rio mais conhecidos, quando se trata de
an�lise formal de protocolos seguros, � o modelo de Dolev-Yao \cite{Dolev1983}. 
Assim, neste modelo, � considerado que a rede est� sobre o dom�nio do
advers�rio, que perante este facto pode extrair, reordenar, reenviar, alterar e
apagar as mensagens que circulam entre quaisquer dois principais legitimos. Com
esta assump��o, entende-se portanto, que o advers�rio transporta a mensagem e
com isso adopta um ataque do tipo \textit{man-in-the-middle}\cite{Stall2005},
com
comportamento incorrecto, que o leva a poder alterar o destinat�rio, atribuir
uma falsa origem, analisar o tr�fego ou alterar as mensagens. Este
funcionamento, entenda-se, n�o � comparado � intrus�o mas sim � intercep��o de
mensagens que pode ser mitigado usando mecanismos de criptografia.

As tipologias de ataque, consideradas pelo o modelo de advers�rio de Dolev-Yao
s�o  instanciadas pela norma X800 \cite{ITU-T1991} que pretende
normalizar
uma arquitectura de seguran�a para o modelo OSI, oferecendo uma abordagem
sistem�tica para o desenho de sistemas seguros. Esta norma considera a seguran�a
sobre tr�s aspectos: ataque, mecanismo e servi�o de
seguran�a\cite{Stall2005}. O
primeiro refere-se � forma usada para comprometer um sistema, por exemplo,
alterando ou tendo acesso n�o autorizado autorizado a dados desse sistema. Na
literatura, algumas vezes usam-se os temos ataque e amea�a para denominarem o
mesmo efeito, no entanto recorrendo ao RFC 2828 \cite{RFC2828} podemos
definir amea�a como uma potencial viola��o de seguran�a, ou seja � apenas uma
possibilidade que pode ser usada para desencadear um ataque explorando uma
vulnerabilidade; no caso do ataque, trata-se da explora��o inteligente de uma ou
mais amea�as que resultam na viola��o com sucesso de um sistema que se pretendia
seguro. O segundo aspecto considerado, na norma X.800, s�o os mecanismos de
seguran�a, que se entende como o processo que permite detectar, prevenir ou
recuperar de uma ataque � seguran�a (ex: encripta��o, controlo de acesso,
assinatura digital)\cite{Stall2005}. Por fim, o terceiro aspecto define os
servi�os
que, fazendo uso de um ou mais mecanismos de seguran�a, permitem resistir a
ataques dirigidos a determinada fonte de informa��o, quer seja durante o
processamento ou durante a comunica��o.
 \subsection{Modelo de Intrus�o em RSSF}\label{sect:subsec_intrusao}
Considerando o estudo de seguran�a numa RSSF, e dada a sua exposi��o
natural, nomeadamente a f�sica, colocando cada sensor ao alcance de um qualquer
advers�rio, torna relevante a considera��o de novos modelos de ataque.
Considerando que cada rede pode ser constitu�da por milhares de sensores, cada
um deles � um ponto de ataque, na impossibilidade de se proteger ou monitorizar
todos os sensores instalados\cite{Perriga}. Assim
as RSSF v�m-se sujeitas a um modelo de advers�rio que difere das redes com/sem
fios convencionais. Um advers�rio pode estar perto da rede e ter acesso aos
sensores e com isto ``roubar'' um ou parte dos sensores da rede com vista a
explorar os segredos ou material criptogr�fico usados para a comunica��o.
Podemos ent�o tipificar estes ataques como sendo por intrus�o. 
Este tipo de ataques podem ser definidos por ataques desde o n�vel
MAC\cite{Xiao2006} at� ao n�vel de intrus�o f�sica  em que um actor externo,
tendo acesso a um ou m�s sensores leg�timos, descobre os segredos criptogr�ficos
permitindo-lhe replicar\cite{Parno2005} os segredos para sensores maliciosos,
que depois de introduzidos podem agir de forma coordenada comprometendo a rede.
Conseguida a intrus�o, o atacante pode induzir nos sensores leg�timos
comportamentos incorrectos baseados na informa��o falsa introduzida pelos
sensores maliciosos, influenciando o processo de encaminhamento (denominados de
ataques ao encaminhamento).  Note-se, por exemplo, que estes ataques t�m
caracter�sticas que os tornam dif�ceis de identificar quando instalados numa
rede,  uma vez que o car�cter aut�nomo das RSSF, torna dif�cil distinguir um
comportamento errado de uma falha. Um sensor malicioso pode respeitar o
protocolo da rede, no entanto podem actuar de forma incorrecta levando a rede a
criar topologias especificas para o ataque (por exemplo, criando parti��es) ou
fazendo, por exemplo, toda a informa��o passar pelos n�s maliciosos, suprimindo
ou violando a informa��o. No que se refere aos ataques direccionados ao
encaminhamento, por serem parte do objectivo do estudo deste trabalho,
encontram-se definidos na pr�xima sec��o e s�o essencialmente instanciados pela
participa��o colaborativa ou isolada de n�s introduzidos  com o intuito de
afectar o normal funcionamento da rede.
\subsubsection{Modelo bizantino: advers�rios bizantinos}

\subsubsection{Novos modelos de advers�rio - solu��es probabilisticas}
As redes de sensores apresentam-se como um 
 \subparagraph{Sum�rio}
    Mediante as vulnerabilidades de uma RSSF, � necess�rio estabelecer um modelo
de advers�rio com vista a poder mapear as capacidades e tipologias de ataques
deste em mecanismos de seguran�a com o prop�sito de lhes poder resistir ou
mitiga-los. O modelo de Dolev-Yao � o modelo de facto quando se trata da an�lise
de amea�as a redes , em que o meio de comunica��o est� sobre controlo do
advers�rio. No entanto, tratando-se de RSSF, este modelo per si n�o se vislumbra
suficiente para abarcar todas as problem�ticas de seguran�a a que este tipo de
redes est� sujeita. Surge assim a necessidade de, face � inseguran�a que cada n�
da rede representa, estender este modelo acrescentando-lhe um modelo de
intrus�o.
    Perante a exposi��o das RSSF, os ataques que se podem desencadear podem ser
diferentes dos observados nas redes convencionais sendo assim necess�rio
considerar outras tipologias de ataques. Assim, podemos classificar os ataques
como activos e passivos \cite{Stallings2005} e os atacantes como internos e
externos\cite{Karlof2003}.
Nestes ultimos ainda se pode classificar quanto aos recursos usados como
\textit{sensor-class} ou \textit{laptop-class}\cite{Karlof2003}. Os ataques que
se consideram para o estudo e
relacionados com as RSSF s�o: falsa informa��o de encaminhamento,
\textit{blackhole},\textit{sinkhole}, \textit{wormhole} e  \textit{sybil
attack}\cite{Douceur2002}.

 
\subsection{Ataques ao Encaminhamento}
Apesar de existirem ataques que podem ser dirigidos a qualquer das camadas da pilha da RSSF, em
particular apresentam-se os ataques relacionados com a camada de rede, respons�vel pelo
encaminhamento de dados. Os protocolo de encaminhamento em MANETs\cite{Corson1999} e em redes de
sensores, de uma forma geral, decomp�e-se em tr�s fases: descoberta dos caminhos, selec��o dos
caminhos e manuten��o da comunica��o pelos caminhos seleccionados. Os ataques a um algoritmo de
encaminhamento, normalmente, podem explorar as vulnerabilidades de cada uma destas fases de forma
especifica. Da�, em seguida se proceder � associa��o dos ataques espec�ficos a cada fase
apresentando as contramedidas que permitem mitig�-los.
\subsection{Ataques � organiza��o da rede e descoberta de n�s} \label{sect:subsec_ataq_org_rede}
A fase de organiza��o da rede e descoberta de n�s � mais vincada em    
protocolos do tipo \textit{table-driven}[REF] uma vez que estes desencadeiam a
cria��o de tabelas de encaminhamento que se dever� manter actualizada durante a execu��o da
rede. No entanto em protocolos do tipo on-demand[REF] tamb�m se verifica a
existencia desta fase mas repete-se em cada inicio de comunica��o ou de
transmiss�o. Assim sendo, um ataque do tipo de introdu��o de informa��o de
encaminhamento falsa, acaba por ter impacto na constru��o da rede e na
descoberta dos n�s. A falsifica��o de informa��o de encaminhamento permite criar
entradas incorrectas nas tabelas de encaminhamento, podendo tamb�m fazer com que
estas fiquem lotadas, isto em protocolos que possuem este mecanismo. No caso dos
protocolos on-demand o impacto � menos efectivo uma vez que o atacante ter� de
estar sempre a injectar informa��o errada a cada inicio de transmiss�o, mas
ainda assim pode provocar danos na rede. 

    Outro ataque efectivo nesta fase � o \textit{rushing attack} [REF]que �
definido pela
explora��o por parte do atacante de uma janela de oportunidade para responder a
um pedido de caminho para um destino. Esta opera��o � desencadeada quando um
protocolo aceita a primeira resposta que recebe (exemplo do AODV[REF]),
calculando isto o atacante � sempre um candidato a ser o pr�ximo encaminhador,
uma vez que n�o respeita temporizadores nem condi��es de resposta.
    Por fim, pode-se ainda tipificar ataques encetados por flooding de mensagens
quer estas sejam  HELLO, para os protocolos que usam este mecanismo para que os
n�s se anunciem aos seus vizinhos, quer sejam mensagens do tipo route-request
(RREQ) para iniciar a descoberta de caminhos. O primeiro permite que um atacante
com maior capacidade de comunica��o se anuncie a todos os n�s como seu vizinho
(atacante laptop-class[REF]) e por isso toda a informa��o possa fluir atrav�s
dele. O segundo, permite explorar as limita��es energ�ticas[REF] da rede levando
os n�s leg�timos da rede a estarem continuamente a responder e fazer as
opera��es inerentes � descoberta de um caminho.
    Para mitigar ataques de HELLO flooding\cite{Karlof1999} pode-se simplesmente
implementar um
mecanismo de acknolege aos pacotes de HELLO. Assim, caso esteja a decorrer um
ataque com uma capacidade comunica��o abrangente de toda a rede, a resposta n�o
ser� possivel porque os n�s mais distanciados n�o possuiram capacidade de
alcan�ar o atacante, anulando por isso o anuncio de vizinhan�a.Outra forma � a
possivel que cada mensagem de HELLO obriga a autentica��o da origem numa
entidade central que ao detectar que um n� se est� a autenticar com esta
mensagem em v�rios n�s ent�o devem ser tomadas medidas para precaver um ataque
desta natureza.

\subsection{Ataques ao estabelecimento de rotas} \label{sect:subsec_ataq_est_rotas}
\begin{description}\addtolength{\itemsep}{-.50\baselineskip}
 \item[\textit{HELLO Flooding }]
Este ataque foi identificado primeiramente por \cite{Karlof2003}  sendo
definido como um ataque que explora alguns protocolos  que se fazem anunciar
aos seus vizinhos pela emiss�o de mensagens de \textit{HELLO}, informando-os da sua
proximidade presen�a\cite{Survey_wsn_Sec_issues}.
Os protocolos que assentam em localiza��o podem ser vulner�veis a este ataque,
uma vez que com um dispositivo do tipo \textit{laptop-class}\cite{Karlof2003}, usando um alcance
r�dio que cubra toda a rede, pode-se anunciar a todos os n�s como vizinho for�ando a
informa��o fluir atrav�s dele.
 \item[Ataque \textit{Sinkhole}]
Nas RSSF um dos modos de comunica��o � de um-para-muitos(\textit{one-to-many}).Este tipo
comunica��o  apresenta alguma vulnerabilidades a ataques do tipo
\textit{sinkhole}\cite{Sinkhole_attack}. Este ataque corresponde
a um atacante informar os n�s vizinhos de dados errados de encaminhamento anunciando-se como um n�
que tem boa comunica��o com o n� sink, tornando-se assim um ponto de passagem de informa��o. O
ataque � realizado enviando pacotes de RREQ, alterando a origem e aumentando o n�mero de
sequ�ncia como forma de fazer garantir que a informa��o se sobrep�e a qualquer outra, legitima, da
rede. Em  determinada altura, um atacante ter� a passar por ele um n�mero elevado de rotas, podendo
alterar ou encaminhar a informa��o de forma selectiva para outros destinos. Os ataques
\textit{table-driven} s�o vulner�veis a estes ataques  enquanto os protocolos baseados em
localiza��o n�o s�o devido �s suas rotas serem \textit{on-demand}.
\cite{Karlof2003,Survey_wsn_Sec_issues,Attaks_defenses_sec_in_wsn}
 \item[Ataque \textit{Wormhole}]
Neste tipo de ataque, apresentado por Perrig et al \cite{Wormhole_perrig} a colabora��o de dois n�s
maliciosos (normalmente a muitos hops de dist�ncia), quer sejam
n�s de \textit{laptop-class}\cite{Karlof2003} ou \textit{sensor-class}\cite{Karlof2003} , contribuem
para uma maior efectividade da ac��o de ataque. Assim, os atacantes estabelecem uma liga��o (ou
t�nel, normalmente de melhor qualidade - maior largura de banda) para comunicarem entre si. Um n�
malicioso captura pacotes ou partes de pacotes e envia-os pela liga��o privada para o outro atacante
para outro extremo da rede.
Este ataque � particularmente eficaz em redes ad-hoc e redes baseadas em localiza��o e sendo estas
compremetidas, n�o conseguiram estabelecer caminhos maiores do que dois hops causando interrup��es
nas comunica��es\cite{perrig_survey_ad_hoc,Survey_wsn_Sec_issues}.
Este ataque transforma o caminho os atacantes em n�s muito solicitados, pois apresentam-se aos
outros n�s participantes como tendo melhor liga��o e a menos dist�ncia do destino.
\cite{WAP_Wormhole}
\item[Ataque \textit{Sybil}]
Este ataque foi definido como um ataque que permitia atingir os mecanismos de redund�ncia
em armazenamento distribu�do em ambientes de ponto-a-ponto (peer-to-peer)\cite{Douceur2002}. Outra
defini��o que surge, agora associada �s RSSF, � a que o define como ``um dispositivo malicioso que
ilegitimamente assume m�ltiplas entidades''\cite{sybil_perrig}. Com estas defini��es e devido � sua
taxonomia � um ataque bastante efectivo contra protocolos de encaminhamento\cite{Karlof2003}. Em
particular dos protocolos que adoptam m�ltiplos caminhos, observa-se ent�o, que um n� ao assumir
v�rias identidades possibilita que na realidade os dados possam estar a passar por um mesmo n�
malicioso\cite{Survey_wsn_Sec_issues,Attaks_defenses_sec_in_wsn}.
\end{description}
\subsubsection{Contra-medidas}
Uma das formas de prevenir um ataque HELLO flooding\cite{Karlof1999} � a implementa��o de
mecanismos de respostas(\textit{aknowlege}) a an�ncios HELLO. Desta forma, caso o atacante esteja a
usar um meio de comunica��o potente, que cubra toda a rede, um n�, em que o atacante se encontre
fora do seu alcance, n�o aceitar� a an�ncio como v�lido.  Para al�m deste mecanismo � poss�vel
proceder � autentica��o da mensagem, certificando-a numa entidade central, que ao detectar que um
n� se anuncia como vizinho de muitos outros n�s, toma precau��es suspeitando que se trata de um
atacante podendo repudiar o n� emitindo uma mensagem para toda a rede\cite{Survey_wsn_Sec_issues}.

Alguns autores t�m vindo a desenvolver algoritmos que visam a detec��o de atacantes que desencadeam
ataques do tipo \textit{Sinkhole}\cite{Sinkhole_attack}, um desses mecanismos � o \textit{Sinkhole
Intrusion Detection Sistem} (SIDS)\cite{Sinkhole_attack} orientado para a detec��o destes ataques
ao protocolo DSR\cite{DSR}. Estes sistema prop�e tr�s mecanismos para detectar um atacante: i)
Discontinuidade de n�meros de sequ�ncia,  tendo em conta que um atacante tentar� usar n�meros de
sequ�ncia muito grandes, por forma a poder fazer prevalecer a sua inform��o, assim um n� pode
identificar os que crescem r�pidamente ou que n�o respeitam uma ordem crescente; ii)Taxa de pacotes
verificados, os vizinhos podem certificar a origem dos pacotes enviados por um n�, isto ser� dificil
de realizar em pacotes de atacantes, uma vez que eles alteram a origem, assim a rede poder� detectar
que est� sobre ataque se circularem muitos pacotes n�o certificados; iii)N�mero de caminhos a
passar por um n�, cada n� pode observar a sua tabela de encaminhamento e se detectar que existem
muitos caminhos a passar pelo mesmo n�, pode desconfiar estar sobre um ataque do tipo
\textit{Sinkhole}\cite{Survey_wsn_Sec_issues,Attaks_defenses_sec_in_wsn}

Alguns autores apresentam mecanismos como a utiliza��o de \textit{packet leashes}
\cite{packet_leashes_perrig} por forma a mitigar o ataque \textit{wormhole}. Preconizam que existem
dois tipos de condi��es para se aceitar os pacotes vindos de uma origem: baseado na localiza��o e
notempo. Assim, o primeiro permite um n� receptor, conhecendo a localiza��o da origem, saber se um 
pacote que atravessou a rede por um \textit{wormhole} calculando a distancia entre os dois pontos.
No segundo caso, baseia-se essencialmente no tempo de transmiss�o do pacote, exigindo ent�o a
sincroniza��o de rel�gios, se for muito r�pido a chegar ao destino, este n� assume que se est�
perante um ataque de \textit{wormhole}.

Para o ataque \textit{sybil} em \cite{sybil_perrig,Survey_wsn_Sec_issues}, s�o fornecidos dois
esquemas de protec��o:
\textit{radio resource testing} (cada vizinho s� pode transmitir num canal, selecciona uma canal
para ouvir, e envia uma mensagem, os n�s que n�o responderem s�o tratados como falsos) e
\textit{random key distribution}.(usando um modelo de \textit{key-pool} s�o atribuidas n keys de
um conjunto de m se dois n�s partilharem q key ent�o podem comunicar de forma segura, existe ainda
uma fun��o de hash, com base no ID do n� para gerar chaves, evitando que um n� possa ter multiplas
chaves)
\subsection{Ataques � manuten��o de rotas}
\label{sect:subsec_ataq_manut_rotas}
%\subsection{Ataques � reorganiza��o da rede } \label{sect:subsec_ataq_reorg_rede}
%tabela_ataques_contramedidas
\subsection{Discuss�o/Resumo}
Como forma de sistematizar a an�lise os ataques ao encaminhamento e as contramedidas para os
mitigar apresenta-se a seguinte tabela resumo.
