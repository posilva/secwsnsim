\section{ Discuss�o e Resumo do Trabalho Relacionado}
As redes de sensores sem fios representam um enorme desafio para a investiga��o de sistemas e
protocolos de seguran�a. As caracter�sticas que as tornam numa mais valia, para a opera��o em
ambientes remotos, apresentam-se como sendo as suas maiores vulnerabilidades em termos de
seguran�a. Este paradoxo � contornado com mecanismos de seguran�a inovadores e que se distinguem
dos existentes nas redes convencionais. Assim, passada em revista as diversas dimens�es que se
pretende abarcar na futura disserta��o:  protocolos de encaminhamento seguro em RSSF e
plataformas de simula��o de RSSF, importa neste momento apresentar uma vis�o critica do trabalho
relacionado como forma de enquadr�-lo como base te�rica da disserta��o.

Em primeiro lugar pode-se apresentar os ataques que foram estudados e apresent�-los, de forma
estruturada, relacionando-os com as contra-medidas para os mitigar.
%tabela de ataques e contra-medidas
\begin{table}[H]
 \centering
\begin{tiny}
\begin{tabular}[t]{l|l|p{4cm}}
\hline
Modelos & Ataque & Contramedidas\\ \hline\hline
\multirow{1}{*}{Dolev-Yao}
& Ataque ao meio de comunica��o & Criptografia sim�trica, \textit{One Way Hashing} \\\cline{2-3}
\hline 
%
\multirow{4}{*}{Organiza��o e Descoberta da Rede}
 & Falsifica��o de informa��o de Routing & Autentica��o, \textit{One Way Hashing}\\\cline{2-3}
%%
 & Ataques de \textit{Rushing} & Selec��o aleat�ria de RREQ, autentica��o, verifica��o bidirectional
\\ \cline{2-3}
\hline
%
\multirow{4}{*}{Estabelecimento de Rotas}
 & HELLO flooding & Autentica��o com verifica��o bidirectional(\textit{acknowledge})\\\cline{2-3}
%% 
& Ataques \textit{Sinkhole} & Autentica��o, Distribui��o de chaves \textit{pairwise} \\\cline{2-3}
%% 
& Ataques \textit{Wormhole} & \textit{Packet leaches}, MAC\\\cline{2-3}
%% 
& Ataques \textit{Sybil} & Distribui��o de chaves \textit{pairwise}, selec��o aleat�ria de canais
de r�dio \\
 \hline
%
\multirow{1}{*}{Manuten��o de Rotas} & Ataques de \textit{Backhole} & Defini��o de temporizadores e
mecanismos de confirma��o (ACK) autenticados\\
\hline
%
\multirow{2}{*}{Modelo de Intrus�o}
& Intrus�o& Encaminhamento multi-rota; \textit{One Way Hashing} \\\cline{2-3}
%%
& Replica��o& Certifica��o central; Autentica��o; N�s vizinhos como testemunhas\\\cline{2-3}
\hline

\end{tabular} 
\caption{Tabela de Ataques \textit{vs} Contramedidas}\label{tab:tabela_ataques_contramedidas}
\end{tiny}
\end{table}

No ponto de vista dos protocolos estudados cabe relacionar as capacidades de cada um para fazer
face a ataques definidos no modelo de advers�rio e tipificados nas diferentes fases dos protocolos
em que estes se podem desencadear.
%tabela de ataques e protocolos de encaminhamento
{%
\begin{table}[H]
\centering
\begin{tiny}
\begin{tabular}{c|c|c|c|c|c|c|c|c|c|}\cline{2-10}
\mc{1}{c}{\textbf{}} & \mc{7}{|c|}{\textbf{Ataques ao Encaminhamento}}&
\mc{1}{|c|}{\textbf{Intrus�o}} & \mc{1}{|c|}{\textbf{Comunica��o}}\\\cline{1-10}
\mc{1}{|c|}{\textbf{Protocolos}} & \mc{1}{|c}{\textbf{Info. Falsa}} &
\mc{1}{|c|}{\textbf{\textit{Rushing}}}& \mc{1}{c|}{\textbf{HELLO flooding}} &
\mc{1}{c|}{\textbf{\textit{Sinkhole}}} & \mc{1}{c|}{\textbf{\textit{Wormhole}}} &
\mc{1}{c|}{\textbf{\textit{Sybil}}}& \mc{1}{c|}{\textbf{\textit{Blackhole}}} &
\mc{1}{c|}{\textbf{\textit{Intrus�o/Replica��o}}} & \mc{1}{c|}{\textbf{\textit{Dolev-Yao}}} \\\hline
%% linhas da tabela
\mc{1}{|c|}{\textbf{SIGF}} & \checkmark & \checkmark &\checkmark & \texttimes &
\mc{1}{c|}{\texttimes} &\checkmark & \mc{1}{c|}{\checkmark} & \texttimes/\texttimes & \checkmark
\\\hline
%
\mc{1}{|c|}{\textbf{INSENS}} & \checkmark & \checkmark & \checkmark & \checkmark  &
\checkmark  & \checkmark & \checkmark  & \checkmark/\texttimes & \checkmark \\\hline
%
\mc{1}{|c|}{\textbf{Clean-Slate}} & \checkmark & \checkmark & \checkmark & \checkmark &
\checkmark & \checkmark& \checkmark & \checkmark/\checkmark & \checkmark\\\hline
\end{tabular}
\caption{Tabela de Protocolos de Encaminhamento \textit{vs} Ataques
}\label{tab:ataques_vs_protocolos}
\end{tiny}
\end{table}
}%

Por fim e sendo a an�lise dos ambientes de simula��o um dos focos do trabalho relacionado
poder-se-� avaliar de forma comparativa os ambientes seleccionados para estudo comparandos com os
crit�rios pensados como adequados para a avalia��o.
\input{tab_simuladores_criterios}