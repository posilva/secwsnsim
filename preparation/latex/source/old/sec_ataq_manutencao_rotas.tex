\subsection{Ataques � manuten��o de rotas} \label{sect:subsec_ataq_manut_rotas}
\paragraph*{\textbf{Ataque \textit{Blackhole}}}
No ataque \textit{blackhole}\cite{HongmeiDeng2002} o atacante intercepta os pacotes destinados ao
n�/�rea que pretende comprometer, informando a origem que este se trata de um caminho de melhor
qualidade. Assim, for�a todo o tr�fego, dirigido ao destino alvo do ataque,  a circular atrav�s
dele. Por exemplo, no protocolo AODV\cite{Perkins1999}, por ser \textit{on-demand} permite que, na
fase de descoberta de uma rota, qualquer n�, que possua um caminho (suficiente recente), responda a
uma mensagem de RREQ. Com isto, este algoritmo de encaminhamento pode ficar sujeito a um ataque
de \textit{blackhole}, pois um n� malicioso interm�dio, pode responder com um caminho melhor,
apesar de n�o ter sequer caminho para o destino, originando um ``buraco negro'', interrompendo o
processo de comunica��o\cite{Survey_wsn_Sec_issues,Attaks_defenses_sec_in_wsn}.
\subsubsection{Contra-medidas}
Para mitigar os ataques de \textit{blackhole} existem v�rias propostas
\cite{blackhole_adhoc,Attaks_defenses_sec_in_wsn,HongmeiDeng2002} das quais se destacam as
que implementam os seguintes mecanismos: i) Confirma��o do destino, � enviada uma mensagens ACK por
cada pacote recibo pelo destino, pelo caminho inverso; iii) Defini��o de limites
de tempo para receber as mensagens de ACK. por parte do destino, ou ao inv�s, receber mensagens de
falha dos n�s interm�dios; iii) Mensagens de falha, quando num n� interm�dio detecta o fim do
temporizador de ACK, este gera uma mensagem de falha; iv) Caminho definido pela origem, ou seja, em
cada pacote � indicado, na origem, o caminho que deve ser seguido pelo pacote at� ao destino.