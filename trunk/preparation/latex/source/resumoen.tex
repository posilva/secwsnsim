\abstract
Sensor networks are an emerging technology in the field of monitoring, independently of physical
environments. They are formed by small devices that self-organize in order to cover a geographical
area can form a network of large scale with thousands of us. This autonomy and self-organization
presents some challenges related to security aspects, in particular, with respect to the routing of
data.

The work undertaken aims to contribute to the creation of a systemic model for the study of secure
routing protocols in sensor networks wireless (WSN). The definition of the type of player is the
initial step in the framework of different types of attack that was assessed. Coupled with the
formal model of Dolev-Yao, which focuses on the attacks on the media, the study of new models of
opponent-related intrusion and capture us is relevant and presented within the context of this work.
    In order to make the WSN resistant to some types of attacks outlined in this type of opponent,
have been developed several routing algorithms
insurance. The aim is to study some of these algorithms, representatives of the state of the art in
this field, establishing an array of measures of resistance to the type of opponent, which then
allows to evaluate the effectiveness of these.

    As a major contribution, this study highlight the design of an innovative simulation
environment, since they intend to implement features not found in simulation systems for the
existing WSN. It will provide the opportunity to design and evaluate routing algorithms are designed
to be safe when subject to attacks in the model defined adversary. This evaluation will focus
primarily on analysis of properties such as energy consumption, reliability, latency, accuracy of
data and correction of the behavior of the Protocol.

% Palavras-chave do resumo em Ingl�s
\begin{keywords}
Wireless Sensor Networks,Secure Routing Protocols,WSN Simulation,Intrusion
Atack
\end{keywords} 
% to add an extra black line
~\\ ~\\ \rule{\textwidth}{0.2mm}
