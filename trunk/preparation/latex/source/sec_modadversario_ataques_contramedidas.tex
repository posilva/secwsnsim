\section{Modelo de Advers�rio, Ataques ao Encaminhamento e
Contra-medidas} \label{sect:sec_mod_adversario_ataq_contramedidas}

S�o v�rios os ataques que se pode direccionar contra a pilha da arquitectura de
uma RSSF. Cada uma das camadas da pilha tem vulnerabilidades pr�prias das
fun��es que desempenha. No escopo deste trabalho o foco � a seguran�a ao n�vel
dos protocolos de encaminhamento, representada pela camada de rede, da�
nesta sec��o se apresentar com algum detalhe as fases tradicionais de um
protocolo de encaminhamento em MANETs\cite{Corson1999} e em particular em
RSSF.\\
Os protocolo de encaminhamento em redes de sensores, de uma forma geral
dividem-se em tr�s fases: descoberta dos caminhos, selec��o dos caminhos,
manuten��o da comunica��o pelos caminhos seleccionados. Importa neste momento
real�ar o facto de que os ataques a um algoritmo de encaminhamento normalmente
exploraram as vulnerabilidades de cada uma destas fases de forma especifica.
Da�, em seguida se proceder � associa��o dos ataques espec�ficos que se podem
desencadear em cada fase e como estes podem ser mitigados aplicando determinados
mecanismos de seguran�a como contra-medidas.
\subsection{Ataques � organiza��o da rede e descoberta de n�s} \label{sect:subsec_ataq_org_rede}
A fase de organiza��o da rede e descoberta de n�s � mais vincada em    
protocolos do tipo \textit{table-driven}[REF] uma vez que estes desencadeiam a
cria��o de tabelas de encaminhamento que se dever� manter actualizada durante a execu��o da
rede. No entanto em protocolos do tipo on-demand[REF] tamb�m se verifica a
existencia desta fase mas repete-se em cada inicio de comunica��o ou de
transmiss�o. Assim sendo, um ataque do tipo de introdu��o de informa��o de
encaminhamento falsa, acaba por ter impacto na constru��o da rede e na
descoberta dos n�s. A falsifica��o de informa��o de encaminhamento permite criar
entradas incorrectas nas tabelas de encaminhamento, podendo tamb�m fazer com que
estas fiquem lotadas, isto em protocolos que possuem este mecanismo. No caso dos
protocolos on-demand o impacto � menos efectivo uma vez que o atacante ter� de
estar sempre a injectar informa��o errada a cada inicio de transmiss�o, mas
ainda assim pode provocar danos na rede. 

    Outro ataque efectivo nesta fase � o \textit{rushing attack} [REF]que �
definido pela
explora��o por parte do atacante de uma janela de oportunidade para responder a
um pedido de caminho para um destino. Esta opera��o � desencadeada quando um
protocolo aceita a primeira resposta que recebe (exemplo do AODV[REF]),
calculando isto o atacante � sempre um candidato a ser o pr�ximo encaminhador,
uma vez que n�o respeita temporizadores nem condi��es de resposta.
    Por fim, pode-se ainda tipificar ataques encetados por flooding de mensagens
quer estas sejam  HELLO, para os protocolos que usam este mecanismo para que os
n�s se anunciem aos seus vizinhos, quer sejam mensagens do tipo route-request
(RREQ) para iniciar a descoberta de caminhos. O primeiro permite que um atacante
com maior capacidade de comunica��o se anuncie a todos os n�s como seu vizinho
(atacante laptop-class[REF]) e por isso toda a informa��o possa fluir atrav�s
dele. O segundo, permite explorar as limita��es energ�ticas[REF] da rede levando
os n�s leg�timos da rede a estarem continuamente a responder e fazer as
opera��es inerentes � descoberta de um caminho.
    Para mitigar ataques de HELLO flooding\cite{Karlof1999} pode-se simplesmente
implementar um
mecanismo de acknolege aos pacotes de HELLO. Assim, caso esteja a decorrer um
ataque com uma capacidade comunica��o abrangente de toda a rede, a resposta n�o
ser� possivel porque os n�s mais distanciados n�o possuiram capacidade de
alcan�ar o atacante, anulando por isso o anuncio de vizinhan�a.Outra forma � a
possivel que cada mensagem de HELLO obriga a autentica��o da origem numa
entidade central que ao detectar que um n� se est� a autenticar com esta
mensagem em v�rios n�s ent�o devem ser tomadas medidas para precaver um ataque
desta natureza.

\subsection{Ataques ao estabelecimento de rotas} \label{sect:subsec_ataq_est_rotas}
\begin{description}\addtolength{\itemsep}{-.50\baselineskip}
 \item[\textit{HELLO Flooding }]
Este ataque foi identificado primeiramente por \cite{Karlof2003}  sendo
definido como um ataque que explora alguns protocolos  que se fazem anunciar
aos seus vizinhos pela emiss�o de mensagens de \textit{HELLO}, informando-os da sua
proximidade presen�a\cite{Survey_wsn_Sec_issues}.
Os protocolos que assentam em localiza��o podem ser vulner�veis a este ataque,
uma vez que com um dispositivo do tipo \textit{laptop-class}\cite{Karlof2003}, usando um alcance
r�dio que cubra toda a rede, pode-se anunciar a todos os n�s como vizinho for�ando a
informa��o fluir atrav�s dele.
 \item[Ataque \textit{Sinkhole}]
Nas RSSF um dos modos de comunica��o � de um-para-muitos(\textit{one-to-many}).Este tipo
comunica��o  apresenta alguma vulnerabilidades a ataques do tipo
\textit{sinkhole}\cite{Sinkhole_attack}. Este ataque corresponde
a um atacante informar os n�s vizinhos de dados errados de encaminhamento anunciando-se como um n�
que tem boa comunica��o com o n� sink, tornando-se assim um ponto de passagem de informa��o. O
ataque � realizado enviando pacotes de RREQ, alterando a origem e aumentando o n�mero de
sequ�ncia como forma de fazer garantir que a informa��o se sobrep�e a qualquer outra, legitima, da
rede. Em  determinada altura, um atacante ter� a passar por ele um n�mero elevado de rotas, podendo
alterar ou encaminhar a informa��o de forma selectiva para outros destinos. Os ataques
\textit{table-driven} s�o vulner�veis a estes ataques  enquanto os protocolos baseados em
localiza��o n�o s�o devido �s suas rotas serem \textit{on-demand}.
\cite{Karlof2003,Survey_wsn_Sec_issues,Attaks_defenses_sec_in_wsn}
 \item[Ataque \textit{Wormhole}]
Neste tipo de ataque, apresentado por Perrig et al \cite{Wormhole_perrig} a colabora��o de dois n�s
maliciosos (normalmente a muitos hops de dist�ncia), quer sejam
n�s de \textit{laptop-class}\cite{Karlof2003} ou \textit{sensor-class}\cite{Karlof2003} , contribuem
para uma maior efectividade da ac��o de ataque. Assim, os atacantes estabelecem uma liga��o (ou
t�nel, normalmente de melhor qualidade - maior largura de banda) para comunicarem entre si. Um n�
malicioso captura pacotes ou partes de pacotes e envia-os pela liga��o privada para o outro atacante
para outro extremo da rede.
Este ataque � particularmente eficaz em redes ad-hoc e redes baseadas em localiza��o e sendo estas
compremetidas, n�o conseguiram estabelecer caminhos maiores do que dois hops causando interrup��es
nas comunica��es\cite{perrig_survey_ad_hoc,Survey_wsn_Sec_issues}.
Este ataque transforma o caminho os atacantes em n�s muito solicitados, pois apresentam-se aos
outros n�s participantes como tendo melhor liga��o e a menos dist�ncia do destino.
\cite{WAP_Wormhole}
\item[Ataque \textit{Sybil}]
Este ataque foi definido como um ataque que permitia atingir os mecanismos de redund�ncia
em armazenamento distribu�do em ambientes de ponto-a-ponto (peer-to-peer)\cite{Douceur2002}. Outra
defini��o que surge, agora associada �s RSSF, � a que o define como ``um dispositivo malicioso que
ilegitimamente assume m�ltiplas entidades''\cite{sybil_perrig}. Com estas defini��es e devido � sua
taxonomia � um ataque bastante efectivo contra protocolos de encaminhamento\cite{Karlof2003}. Em
particular dos protocolos que adoptam m�ltiplos caminhos, observa-se ent�o, que um n� ao assumir
v�rias identidades possibilita que na realidade os dados possam estar a passar por um mesmo n�
malicioso\cite{Survey_wsn_Sec_issues,Attaks_defenses_sec_in_wsn}.
\end{description}
\subsubsection{Contra-medidas}
Uma das formas de prevenir um ataque HELLO flooding\cite{Karlof1999} � a implementa��o de
mecanismos de respostas(\textit{aknowlege}) a an�ncios HELLO. Desta forma, caso o atacante esteja a
usar um meio de comunica��o potente, que cubra toda a rede, um n�, em que o atacante se encontre
fora do seu alcance, n�o aceitar� a an�ncio como v�lido.  Para al�m deste mecanismo � poss�vel
proceder � autentica��o da mensagem, certificando-a numa entidade central, que ao detectar que um
n� se anuncia como vizinho de muitos outros n�s, toma precau��es suspeitando que se trata de um
atacante podendo repudiar o n� emitindo uma mensagem para toda a rede\cite{Survey_wsn_Sec_issues}.

Alguns autores t�m vindo a desenvolver algoritmos que visam a detec��o de atacantes que desencadeam
ataques do tipo \textit{Sinkhole}\cite{Sinkhole_attack}, um desses mecanismos � o \textit{Sinkhole
Intrusion Detection Sistem} (SIDS)\cite{Sinkhole_attack} orientado para a detec��o destes ataques
ao protocolo DSR\cite{DSR}. Estes sistema prop�e tr�s mecanismos para detectar um atacante: i)
Discontinuidade de n�meros de sequ�ncia,  tendo em conta que um atacante tentar� usar n�meros de
sequ�ncia muito grandes, por forma a poder fazer prevalecer a sua inform��o, assim um n� pode
identificar os que crescem r�pidamente ou que n�o respeitam uma ordem crescente; ii)Taxa de pacotes
verificados, os vizinhos podem certificar a origem dos pacotes enviados por um n�, isto ser� dificil
de realizar em pacotes de atacantes, uma vez que eles alteram a origem, assim a rede poder� detectar
que est� sobre ataque se circularem muitos pacotes n�o certificados; iii)N�mero de caminhos a
passar por um n�, cada n� pode observar a sua tabela de encaminhamento e se detectar que existem
muitos caminhos a passar pelo mesmo n�, pode desconfiar estar sobre um ataque do tipo
\textit{Sinkhole}\cite{Survey_wsn_Sec_issues,Attaks_defenses_sec_in_wsn}

Alguns autores apresentam mecanismos como a utiliza��o de \textit{packet leashes}
\cite{packet_leashes_perrig} por forma a mitigar o ataque \textit{wormhole}. Preconizam que existem
dois tipos de condi��es para se aceitar os pacotes vindos de uma origem: baseado na localiza��o e
notempo. Assim, o primeiro permite um n� receptor, conhecendo a localiza��o da origem, saber se um 
pacote que atravessou a rede por um \textit{wormhole} calculando a distancia entre os dois pontos.
No segundo caso, baseia-se essencialmente no tempo de transmiss�o do pacote, exigindo ent�o a
sincroniza��o de rel�gios, se for muito r�pido a chegar ao destino, este n� assume que se est�
perante um ataque de \textit{wormhole}.

Para o ataque \textit{sybil} em \cite{sybil_perrig,Survey_wsn_Sec_issues}, s�o fornecidos dois
esquemas de protec��o:
\textit{radio resource testing} (cada vizinho s� pode transmitir num canal, selecciona uma canal
para ouvir, e envia uma mensagem, os n�s que n�o responderem s�o tratados como falsos) e
\textit{random key distribution}.(usando um modelo de \textit{key-pool} s�o atribuidas n keys de
um conjunto de m se dois n�s partilharem q key ent�o podem comunicar de forma segura, existe ainda
uma fun��o de hash, com base no ID do n� para gerar chaves, evitando que um n� possa ter multiplas
chaves)
\subsection{Ataques � manuten��o de rotas}
\label{sect:subsec_ataq_manut_rotas}
%\subsection{Ataques � reorganiza��o da rede } \label{sect:subsec_ataq_reorg_rede}
%tabela_ataques_contramedidas
\subsection{Discuss�o/Resumo}
Como forma de sistematizar a an�lise os ataques ao encaminhamento e as contramedidas para os
mitigar apresenta-se a seguinte tabela resumo.
